\section{Reflection}
There may be some ethical concerns surrounding the use of automation in regards to job applications.
Will it make us less honest? Will it create an unfair advantage?\\
As outlined in the software section of the analysis: Aiding once ability to be productive, is not unethical, as long as it's not lying.
Just as having the ability to use spell checking, didn't make us more unethical as a species, but to the contrary: More productive.
It won't give an unfair advantage, as it's a mean to be more productive, not a mean to create CV's of greater quality. If ones skills sucks, the program won't help much.
In fact, we would argue that it levels out the playing field, since those who may need our program, are those who have the least time, but may be just as qualified. Again, the program is a tool, not a magic solution.\\

One might say, that creating unethical means of getting an interview encourages unethical means of filtering people out.
But the encouragement to use unethical means of filtering people out already exists:
It's called profit and cost management. By the very nature of capitalism, the companies are encouraged to be unethical.
If one wishes to advance, we need not only successful companies, but also successful people.
Adding unto this, we need not the few to be successful, we need the society at large to be successful. This program encourages not unethical behavior, but more so equality of opportunity.
As those qualified people that are bad at applying, may still end up qualifying for the same position as those whose specialty is the skill in applications.
\newpage