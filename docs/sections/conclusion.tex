\section{Conclusion}\label{sec:conclusion}
This project’s overarching objective was to aid The United States and The United Kingdom unemployed in obtaining a job by means of CV. 
The target group of the unemployed was chosen due to the findings that unemployment had a negative effect on the economy of the US and UK.
Furthermore, it was found that unemployment drastically reduces the likelihood of obtaining a job as time goes on.
Therefore, to aid the unemployed in obtaining a job, the process behind applying and obtaining a job by CV was discussed. 
It was determined that there were five stages to obtaining a job and that the two most paramount were the ATS scanner and hiring personnel.
Therefore, the three elements that the ATS scanner scans for in a CV and the three concepts that would pass the hiring personnel´s “7-second test”. \\

Using our previous findings, the factors which determined the likelihood of an unemployed obtaining a job were found. 
These factors were then analyzed and it was found that the most productive method of helping the unemployed was to optimize some of these factors.
The factors which optimized would benefit the unemployed most were found to be; sending a large quantity of CVs, passing the ATS scanner and passing the hiring personnel. \\

To optimize sending a large quantity of CVs, it was determined that using a software solution would be ideal. 
To optimize passing the ATS scanner, keywords would be incorporated into the unemployed applicant’s CV, the CV headers would be correctly named and the CV would be produced as a PDF by LaTeX. 
To optimize passing the hiring personnel, the CV would be aptly structured and streamlined. 
A software solution would further assist in systematically incorporating the optimizations into a CV, to pass the ATS scanner and hiring personnel. \\

These optimizations became a C program where a user, likely an unemployed applicant, would write a “Long CV” consisting of all their qualifications.
The user would thereafter use keywords to streamline the Long CV into a relevant CV.
The users would also write their work experience and education and this would be structured efficiently and lastly be converted into a PDF, ready to be used in the unemployed's application process. 
Simply put, this project objective became a program that automatically generates a digital CV optimized for the US and UK unemployed. \\

This project’s digital CV was thereafter tested by a job counseling expert, which should simulate the hiring personnel. 
A CV produced by the program was compared to a CV written entirely by human hands.
The feedback for the program passing the hiring personnel were good, but still lacking. 
This was due to the fact that the program’s free text lacked a smooth transition. Though, the contents of the CV was a drastic improvement.

The original problem statement seems to have been addressed and resolved on the lowest level, though many improvements are necessary to make this applicable for the mainstream.
\clearpage