\section{Conclusion}\label{sec:conclusion}
This research aimed to identify effective ways of finding a solution for a better digital CV.
Based on the research of making a CV, a lot of time and effort can be consumed by it,
and nevertheless achieve nothing at the end.
These quantitative data has proven that not many of the applicants can make it past the ATS scanner 
even when sending a lot of CVs to different companies.
Companies has different criteria of a CV that means they have different keywords. 
There have been a lot to research for reaching a narrowing of the problem, 
but it has come to this that three software solution proposals.
Structure and streamlining played also role, so the text can be more reliable for readable after getting past the ATS scanner
and keyword matching is good for job posting's criteria. \\

Based on the problem statement of enabling British and American unemployed people to create a large quantity of CV applications 
that can get a higher chance of getting passed the ATS scanner and hiring personal 
if they were to individually create each application, while expending little effort, 
these has been made out to be a c program where the solution is a CV generator with LaTeX format. \\

This product has been tested out with an job counseling expert and therefore has experience with CVs. 
The result isn't necessarily perfect for reaching the goal of the problem statement, 
and this software solution is done in a way that bullet points is everywhere compared to a self-written CV.
In comparison of the knowledge this has been a struggle to prove and the text doesn't have a red thread to give a smooth transition. \\

In terms of this solution compared to other solutions this can be only used for minor job postings, 
because it lack the ability to write properly in some sections, 
still it is important to note that it is a solution to strike against quantity 
and a has higher chance of getting passed the ATS scanner.


\clearpage