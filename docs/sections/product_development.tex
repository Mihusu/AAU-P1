\section{Product Development}\label{sec:product_development}
\subsection{Planing phase}
Before building the program, we brainstormed and created a Minimal Viable Product (MVP).
This MVP (which can be seen illustrate beneeth) is a general high level schematic outlining what our program is supposed to do.

\incfigure{figures/engMVP}{fig: MVP}{MVP}
Here we discussed what are the most essential parts, and came to the conclusion that the following things
were needed: The Users skills/abilities and the information from the vacant job position. These things
should be transformed into a custom application. 

After further consideration, we had encountered a problem, that generating a custom CV from nothing but
skills/abilities would create a very childlike, maybe even unreadable, application. This could be solved with a lot
of data, and therefore a lot of time calibrating the process of automatically creating sentences. This may also result
in a lower quality application, if this calibration doesn't happen. 
A CV, that might and might not even get through the "ATS/keyword scanner" outlined in the analysis. 

We therefore sought to instead create a filter, where the high quality sentences would be somewhat guaranteed and maintained.
This filter is only supposed to filter out all the unneccessary parts of a longer quality CV, into an application that
is ready to be sent. In other words, we decided to instead concentrate on creating a tool that aids
in sending out applications, instead of a CV generation program.

We then changed the MVP schematic to the following:
\incfigure{figures/engMVP}{fig: MVP}{Revised MVP}
This program is supposed to take in the keywords/job application, some formel requirements for
the structure of the new CV, and the originally long CV. 

The long CV should contain as many pages as possible, of all abilities and skills one has.
It should also include prior work experience, and anything else relevant on a CV. 

In this way, the program chooses the most important sentences, work experience and skills to be included,
and from the structural requirements creates a new CV.

From here we decided to organize our MVP into a UML diagram, to make it more apparent
what the essential functions were supposed to do, and which functions that were essential:
\incfigure{figures/UML}{fig: UML}{UML 1.0}
