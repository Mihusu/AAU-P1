\section{Product Development}\label{sec:product_development}
\subsection{Planing phase}
Before building the program, we brainstormed and created a Minimal Viable Product (MVP).
This MVP (which can be seen illustrate beneeth) is a general high level schematic outlining what our program is supposed to do.

\incfigure{figures/engMVP}{fig: MVP}{MVP}
Here we discussed what are the most essential parts, and came to the conclusion that the following things
were needed: The Users skills/abilities and the information from the vacant job position. These things
should be transformed into a custom application. 

After further consideration, we had encountered a problem, that generating a custom CV from nothing but
skills/abilities would create a very childlike, maybe even unreadable, application. This could be solved with a lot
of data, and therefore a lot of time calibrating the process of automatically creating sentences. This may also result
in a lower quality application, if this calibration doesn't happen. 
A CV, that might and might not even get through the "ATS/keyword scanner" outlined in the analysis. 

We therefore sought to instead create a filter, where the high quality sentences would be somewhat guaranteed and maintained.
This filter is only supposed to filter out all the unneccessary parts of a longer quality CV, into an application that
is ready to be sent. In other words, we decided to instead concentrate on creating a tool that aids
in sending out applications, instead of a CV generation program.

We then changed the MVP schematic to the following:
\incfigure{figures/Program_process_diagram}{fig: MVP}{Revised MVP}
This program is supposed to take in the keywords/job application, some formel requirements for
the structure of the new CV, and the originally long CV. 

The long CV should contain as many pages as possible, of all abilities and skills one has.
It should also include prior work experience, and anything else relevant on a CV. 

In this way, the program chooses the most important sentences, work experience and skills to be included,
and from the structural requirements creates a new CV.

From here we decided to organize our MVP into a UML diagram, to make it more apparent
what the essential functions were supposed to do, and which functions that were essential:
\incfigure{figures/UML}{fig: UML}{UML 1.0}

\subsection{Read.c}
\subsection{Filter.c}
%Indsæt filter strukturen som billede

%talk about what this part should do

\subsubsection{Removing personal pronouns}
After receiving the "free text" part, from the read.c file, we need to remove any personal pronouns.
We do this, since personal pronouns are quite unneccessary, since they don't add any substance to the sentences.
This is important to do since recruiters, as we talked about in the background section, spend a fraction of a minute
to look through the application. Having more unneccessary words will just distract the recruiters from understanding why they should hire you.
As talked about in the background section, not using personal pronouns increases your
chances of getting an interview by 55 percent.

To remove all personal pronouns before filtering the rest of the text, we made the following function:
\incfigure{figures/personal_pronoun}{fig: pp function}{remove_personal_pronouns}

It's a pretty simple function. To start it reads a certain section. Then it removes all personal pronouns at the start of each sentence in that section.
It loops through it, until there is no longer any personal pronouns left.

In order to identity which words are personal pronouns, we have hardcoded a list of personal pronouns (including some words that are also unneccessary), that will be 'case insensitive compared'.
Each time a word is identified as a personal pronoun, a loop swifts all the words 1 to the left in the array, where the last word in that array will then be freed.

There are also some hardcoded punctuation symbols, to indicate what is defined as the "start of a sentence". 

this results in the following:
\incfigure{figures/personal_pronoun_ex}{fig: pp example}{Removing Personal Pronouns}

As we can see, the filtered result is much cleaner and easier to read, without losing any information.
There is only problem left with it. Since the words are just shifted 1 to the left in the array, the sentence no longer begings
with a capitalized letter. This will later be fixed in a function in format.c, where such things will be formated.

\subsubsection{}
\subsection{Format.c}
\subsection{CV-Gen.c}
\subsection{Makefile}