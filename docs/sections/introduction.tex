\section{Introduction}\label{sec:introduction}
Every country around the world rely on its citizens being productive, this usually translates to a job. 
It is therefore costly when unemployment happens and when those that might have much to contribute, never get the chance. 
This is partly due to the difficulty of both applying and obtaining a job, specifically for the unemployed. 
To ratify this, the overarching objective of this project is to aid The United States and The United Kingdom unemployed in obtaining a job using a CV. \\

To fulfill this objective, the process of obtaining a job, along with the content and structure of the CV will need to be discussed.
Furthermore, the unique aspects of applying as unemployed will need to be outlined so to better aid them in obtaining a job.
Using what has been discussed, it will then be possible to outline many of the crucial factors that determine the chances of the unemployed obtaining a job. \\

Analyzing these factors, the most relevant factors to the unemployed will be selected.
It will then be theorized how the selected factors could be optimized to increase the chances of the unemployed obtaining a job. \\

Utilizing the theoretical optimizations, it will be attempted to implement these optimizations into a product.
The creation of the product will be documented and any decisions taken examined and discussed. 
The product will be tested in practice and the results from the test document will be analyzed.
The effectiveness of the product will then be evaluated, to answer if the project's following problem statement was solved: "How can we make a software solution, that enables the British and American unemployed people, 
to create a large number of CV applications. \\

Such that they can get a higher chance of getting past the ATS scanner and hiring personal, than
if they were to individually create each application while not expending more effort?"
Thereafter, any further development of the product including ethical concerns will be examined, discussed, and reflected upon. \\

\newpage

