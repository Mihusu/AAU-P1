\section{Discussion}\label{sec:discussion}
Even though the program works much to the intent, it still leaves a multitude of things to be desired:
If two sections has the same content, but formulated in different ways, both sections will be included.
This makes the finished product seem unprofessional at best. As an example, with the keyword "English", and the sentences:
\begin{itemize}
  \item 1. "Fluent in English"
  \item 2. "Mastery level of English grammar"
\end{itemize}
Both mean approximately the same, but only one sentence should be included. For this, one could envision a system, where only the section that matches "English" with the highest density, should be included.\\

Another problem stems from the fact, that the output can seem disjoint, cold and robotic. This could be fixed, if the filter was capable of understand basic semantics, and filling in conjunctions that make the output more cohesive.\\

The structure, depending on which industry, may not always be the best suited structure for the job position. If one was to develop further, this will surely have to be more dynamic.\\

Then there is the problem, that most specialized high level job openings require both a CV and a separate job application. This program will be effectively useless in the cause to aid the job applicant in this process.
But this program is much more of a proof of concept, then a ready to go fix all solution.
Though most people who are educated enough to apply for such job positions are probably able to write it them selves without much trouble.
In the end, the experiment showed that this solution is much more applicable, to those who have a hard time narrowing what skills and abilities should be included.
To the people who are generally more desperate and "just need a job", as a lot of unemployed are.
In this instance, the product we have designed would be viable.\\

There is also the problem, that the program is restricted to linux, which is not ideal, seeing as a larger majority of people use Windows.
Even if you have linux, and follow the steps outlined in the readme file, then it's still a hazel to actually make it work, if one doesn't have a technical background.
If the program had a GUI instead, or run on a website such that no extra programs needed to be installed, then it might be viable for deployment.
Until these things have been fixed, then the program is only viable as a niche solution, as it's to underdeveloped to catch mainstream success.