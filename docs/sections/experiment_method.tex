\section{Method}\label{sec:method}
\subsection{Quantitative test}
In order to test if our product helps solve our product statement, we wish
to make an experiment that quantitatively evaluates the effectiveness of your
program. 

In order to do this, we have devised the following experiment:
\incfigure{figures/experiment1}{fig:Experiment}{Method}
In the experiment we will create a long CV where all the original information
is in. We do this to ensure that there is no extra information in the human 
rewritten CV that our program couldn't have created. In this way, we ensure
the test is as fair as possible. 

The human will shorten down and create a CV based on the long CV. This CV is
is gonna be sent to n number of job postings.
Our program, is gonna filter down the long CV, and create a job application,
for each individual job posting. 
To ensure, that some job postings don't just have lower requirements, we will be
sending the new CV applications from both the program and the person to the same 
job postings each time. In total we will be sending out n number of applications.

In this way, we can assume that both the human who had rewritten one
CV and the program, who has rewritten n number of CV's,
will take about the same amount of time, to send out n CV's.
Therefore we can somewhat ensure the results as a function of effort are gonna
be fair.

This experiment will compare the results of how many times one gets a job
interview using one of the two methods: Comparing the amount of 
job interviews from the qualitatively created CV from
the human, to the quantitatively created CV from the program to
each other.

The results from this, will be a good indication, whether our problem solved, or
atleast helped, in the process of getting a job interview, or if just manually
creating one good interview is better.
