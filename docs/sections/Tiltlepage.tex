{\selectlanguage{english}
\aautitlepage{%
  \englishprojectinfo{
     P0-Projektet %title
  }{%
    CV - generation valg %theme
  }{%
    Efterårssemestret 2020 %project period
  }{%
    SW 1 Gruppe 005 Klynge 1 % project group
  }{%
    %list of group members
    David Doctor Heyde Rasmussen \\
    Hans Erik Heje\\ 
    Mikkel Kaa\\
    Ming Hui Sun
  }{%
    %list of supervisors
    Johannes Bjerva
  }{%
    18. december 2020 % date of completion
  }%
}{%department and address
  \textbf{Software}\\
  Aalborg Universitet\\
  \href{http://www.aau.dk}{http://www.aau.dk}
}{% the abstract
  This project is going to focus on the theme "Digital CV" 
  which this project will find a problem for those people who struggle with making a CV,
  and the reasons behind the rejection. It is well-established that networking is very common to get a job in most countries, 
  and even when people are sending a lot of CVs to different companies 
  there is still a good chance of it being rejected by all of them.
  The project aims to find a software solution for those who are in need of a better CV 
  and a better chance for getting past the ATS scanner and hiring personnel.\\
  To test if the software solution was a success, 
  a job counselor was assigned to check one CV from a human and 15 different CVs from a computer with 15 different job postings; comparing them
  to determine which of these were most likely to get an interview from the specific job position. 
  The feedback concluded that structure and contents were essential to get a good result and only one of them did the program succeed in.
  Lacking the adequate capabilities to structure  CV as fluently as a human, might deter the hiring personal. Despite this, the ATS scanner was a success 
  because of the final products filtering capabilities.
}}