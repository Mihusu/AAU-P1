{\selectlanguage{danish}
\aautitlepage{%
  \danishprojectinfo{
     P0-Projektet %title
  }{%
    Elektroniske valg %theme
  }{%
    Efterårssemestret 2020 %project period
  }{%
    SW 1 Gruppe 005 Klynge 1 % project group
  }{%
    %list of group members
    David Doctor Heyde Rasmussen \\
    David Nikolaj Vinje\\
    Hans Erik Heje\\ 
    Kristoffer Bach Wilhjelm\\
    Mikkel Kaa\\
    Ming Hui Sun
  }{%
    %list of supervisors
    Anders Schlichtkrull
  }{%
    30. september 2020 % date of completion
  }%
}{%department and address
  \textbf{Software}\\
  Aalborg Universitet\\
  \href{http://www.aau.dk}{http://www.aau.dk}
}{% the abstract
  This paper is about electronic voting, and what it can do for the whole world, but there's more to it than an effective system, and right now many countries are using it, because of economic benefits. It's also quicker to count all the votes. The paper contains of requirements for good and secure voting system, and it explores  the benefits and consequences of electronic voting, compared to its more traditional counterpart: The paper ballot. Some countries which are behind the newest technology can be more in danger than other, as well as small countries with less people in the software and hardware industry. As the technology is being developed so will the security, anonymity, transparency. Delving deeper into how it might shake up the elections for either the worse or the better, and how one of the more significant flaws might be counteracted will take time to produce.
}}