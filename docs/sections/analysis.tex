\section{Analysis}\label{sec:analysis}

\subsection{Software solutions}
A software solution can be many things where it can go from keyword matching to print a pdf file for a CV.
More realistic CV can take it to a balanced level that are not very ambitious, 
so, the employer are more interested to have you. Every time the CV needs to be edited
then the applicant need to use the time to be quick and efficient, and afterwords it can be reuse.
Simply put software solutions will make it easier for applicant to apply for a job instead of using a long time to edit and write. 
Earlier in the text there were written some information 
that quantity and quality of a software solution will be the most optimal for a situational content,
but it depends on which people will be using it, because low educated have a more tendency to apply
in a wide area of jobs without thinking too much about the category. 
Even though quantity and quality contents would be great for the applicant, 
there can still be doubts for the implemented software, since a high educated CV can have many standardized format and information,
and therefore, it can be a good idea to check it one more time instead of using a software one time and then sent it to the employer.
People with a higher education think differently since they want a specific job
with a good monthly salary and the motivation to work there. 
Above all these software solutions can provide help if quality is not very important factor,
and that will be the pros. For the cons that will be less efficient for people with a high education.


\subsubsection{Streamlining}
If people are applying for a apprenticeship, job or traineeship, 
then it can be a good idea to streamline the CV in the future, because employers don't like non relevant information.
There can be scenarios for a bad CV, witch could be for example a employer got a lot of CVs,
and that would be terrible if he/she should read a long CV. It would be better to sort out only the most important information 
like the education, work experience, and a good and short resume text. 
these scenarios with a eight-page cv can be immediately turn down by the employer or it could be sorted out of the qualified list.
That's why it can be a good idea to keep it short, since it is the most effective way of not letting the employer to read a long time.

Try to see the employer's perspective for hiring. Every employer have different expectations, 
and that is where the streamlining begins. Streamlining has multiple pros as an software solution,
for example it does only need a few user input, so it will automatically create a cv with a sorting method. 
It's better to use valuable space for something than a mass of non relevant texts.
Promotes skills and experience can help a lot, but writing can take time, and streamlining can help to do it quicker 
and print out at last a pdf file automatically, so it is ready to use.
There is a chance that all of an applicant's experience does not necessarily contribute to the CV, 
because the employer is not interested the positions he/she had before, they want to know what the person have achieved
or the contributions the applicant have made. 


\subsubsection{Formatting/producing a CV} {
Sorting a CV is an important factor for having the CV to at least getting through the qualifying test also known as (ATS),
and that is one way of a keyword matching a CV. There are still how to format a CV, 
while doing that it can be frustrating to do a specific design on a CV for every job, 
but most of the time there are standard types of CVs.
So a software solution to format a CV can be useful for design the content, contact information and a short resume,
and there are also other factors in place to an automatically format generator. Again they are almost no user input to be needed, 
so the amount of time using to format a CV is pretty short. 

}