\section{Analysis}\label{sec:analysis}

\subsection{Implications and process}
For a digital CV program to be relevant, it has to make the process of applying for a job by mean of CV more effective.  
It is therefore essential to analyze the implications and process of applying to a job by means of CV.

Firstly, the implications of applying to a job constitutes that one is in need of a job and therefore needs a CV.
That means those could benefit from a digital CV is anybody in need of a job, or searching for a better job. 
The most obvious benefits are those with no job in the first place. 
The unemployment rates vary from country to country and can shift rapidly.
As discussed in background(Unemployment rates), Denmark, a small and generally economically robust country, has 137800 unemployed according to the newest figures. 
This means a potential 137800 Danish people could benefit from a more effective CV process. 
When looking at a larger country such as the United States, that number is 12.6 million.
That is a massive potential beneficiary group, it is therefore in the interest of any nation to encourage a more effective CV process.
The potential for a digital CV to help those that are unemployed is elevated when one considered the time constraint for most unemployed.
Being unemployed, especially for longer periods of time, can have a severe impact on the likelihood of ever getting a job.
Australian Council of Social Service CEO Cassandra Goldie points to the fact that after a year of unemployment, "you almost halve your chances of ever getting back into employment".

As outlined in background(The numbers behind the process of getting a job), there is only an 8.3 percent chance that an applicant, unemployed and in transition, will be invited for a job interview.
Furthermore it takes 10-15 interviews before an applicant gets a job.
In the end most applicants need to send an estimated 150 job applications in addition to the job interviews, before getting a single job offer.
Therefore it is especially vital for those that are unemployed to apply to as many jobs and possible within a small time frame. 
This time frame is further threatened when you consider the time it takes for hiring personnel to view and respond to your application.
This is heavily dependant on what job one is applying for, most hiring personnel will respond within 3 days. 
However, for less in demand labour or unskilled labour it can take anywhere from 10 to 30 days. 
This is in the case if the hiring personnel actually responds to the applicant, which in some cases they don't bother to at all.
For an unemployed applicant it takes on average 22 weeks and sometimes hundreds of CV´s to find a job. 
With the large quantities of CV´s needed to get a job and the time constraint of becoming less likely to be employed as time goes on.
The unemployed population have much to gain from a program able to produce a large quantity of CV´s quickly.

Those that are already in employment can also gain much from a more effective CV process.
However, as shown in the background figure one (The numbers behind the process of getting a job), 80 to 95 percent have gained there job not actively searching.
In these categories we can see that on average 61 percent of them gained there job through networking. 
This points to the fact that in most job cases a more effective CV process will not help.
This raises questions about the actual usability of a more effective CV process, as such a process will not change the fact that networking for a job in most cases will be superior to applying.
However, when we look at the active candidates, we see that 40 percent of them gained there job through applying. 
This means that they could benefit much more from a more effective CV process.
On a whole it could be said that if the intention of the project was to make the process of getting a job more effective for anybody and everybody, it would make more sense to make a program that made networking easier.
However, as everybody is unemployed at some point and that those that are unemployed theoretically produce nothing for their society:
The simplest and quickest way to benefit society as a whole and the individuals that are unemployed is to make an effective CV process that enables them to get a job and carry there own weight in our society.
%https://www.linkedin.com/pulse/new-survey-reveals-85-all-jobs-filled-via-networking-lou-adler/

When applying to a job by means of CV, different job positions advise different required and situational content for the CV.
As outlined in background(Different expectations in a company), over 90 percent of all positions in the digital world will have a screening process.
Most firms will do this through an "Applicant Tracking System" (ATS), which will scan the CV for keywords that match the applicant to the position.
Furthermore, according to Caitlin Proctor it is stated that "Nearly 75 percent of resumes are rejected because they’re not correctly formatted or keyword optimized."
This means that formatting your CV correctly to appease the scanning systems could make an essential difference in an applicants hunt for a job.
A program that uses keywords to build a CV is therefore useful as it passes the scanning systems, unlike a CV that is not structured to pass the scanning.
This structuring through keywords has another purpose as well.
James Reed is the author of the book "The 7 Second CV", he puts much emphasize on the structure, relevance and wording of a CV.
The concept is that generally most hiring personnel will only spend 7 seconds on each CV and that a CV must be able to stand out in those first 7 seconds.
This can be solved with a digital CV program if it can find a standard structure and content that always passes the 7 second test.
https://www.glassdoor.co.uk/blog/7-second-cv/

\subsection{Solutions}
One of the greatest problems with the process of job hunting, is just how impersonal
the recruitment process is. As stated earlier, one of the most impersonal aspects
of finding a job is that using the wrong keywords can very
quickly land your CV in the spam folder.
Even if your CV gets through, it might be rejected and thrown away because of
minute errors, since the recruiter can't judge you as a person, through the CV.
This fact can easily be turned around to your advantage, as you can optimize the
job hunting process in three steps:   
\begin{itemize}
  \item Get through the keyword detection algorithm
  \item Optimize the quality of the CV
  \item Send a large quantity of CVs
\end{itemize}

It is easily seen, that the the stiffer the competition is, the more one 
needs to focus on the first aspect. This is evident by the fact, that a 
recruiter only wants a few of the most relevant applications.
The more specialized one's role is, the more one needs to focus on the second
aspect. This is evident by the fact, that not many will be eligible to qualify
for an advanced specialized role.
The more general the job is, the more you need to focus on the third
aspect. This is clearly seen, as the more general the job is, the more can 
qualify to fill that role. This makes the job-hunting process more about luck, 
as almost anybody will have the qualifications for the job.
Luckily, luck is nearly a function of probability, which can be optimized using
quantity.

This makes a software solution an optimal way of making the jog hunting process more effective,
since all three of these aspects can drastically be improved and automated
using specialized software. 
The first step can easily be filtered through, to translate all relevant, but
misspelled or incorrectly worded, keywords into their correct form. Such
as the word "cocktail expert" into "bartender" using software.

The second point can be fixed through many thousands of data points, using
artificial intelligence, to aid how to formulate oneself properly. 

The third step is properly what software is best at: Repetitive and iterative
tasks.
One could automate the filtering of irrelevant parts of a CV out, and then send
many thousand of pseudo specialized CVs out without requiring much work from the
user.

Since we want to reduce the general unemployment rate, the solution is one that needs to 
be as general as possible. Therefor fixing 'step one' is essential.
From there, we could create different tools, to help one create the most relevant
CV for the situation the applicant is in.

The quality aspect (step two) is more important for more specialized roles, 
but since the intent is to help as many people as possible then the solution
must be a aiding program, and not a writing program.
This is the case, since specialized roles are far and few in between,
as such, 'step two', is not the biggest issue, for most people. 

'Step three' is also of a greater importance, since being able to create a
greater quantity of applications, drastically increases one's chance of
employment. 
As such having a software solution be able to automate all the most boring
bits of writing applications is incredible useful.

Some people wish to spend time on writing applications, but either have
disabilities or other issues making it hard to send an appropriate amount
of applications out. Having a program to automate most of the process, enhances
the challenged peoples' ability to acquire a job.

It could be argued, that having everyone use aiding software to enhance ones
quantity of applications would eventually even out the playing field, making
it such that we have the same problem as before. 
But we do not think this is the case, since this kind of software only seeks
to eliminate the many firewalls recruiters use to filter people out, be it 
qualified applicants or not.
This, in turn creates a playing field  where one is not determined by
mistakes that have little to do with the job, but hopefully more on one's merit. 
Since recruiters will have to individually sort through bad applicants, instead of having 
a program that sorts all the applicants with simply mistakes out. In the worst-case scenario, more
people would have enough energy to send an appropriate quantity of applications
out, and recruiters will have more people to choose from, in turn increasing
equality of opportunity.

\subsection{Software solutions}
A software solution can be many things where it can go from keyword matching to print a pdf file for a CV.
More realistic CV can take it to a balanced level that are not very ambitious, 
so, the recruiter are more interested to have you. Every time the CV needs to be edited
then the applicant need to use the time to be quick and efficient, and afterwords it can be reuse.
Simply put software solutions will make it easier for applicant to apply for a job instead of using a long time to edit and write. 
Earlier in the text there were written some information 
that quantity and quality of a software solution will be the most optimal for a situational content,
but it depends on which people will be using it, because low educated have a more tendency to apply
in a wide area of jobs without thinking too much about the category. 
Even though quantity and quality contents would be great for the applicant, 
there can still be doubts for the implemented software, since a high educated CV can have many standardized format and information,
and therefore, it can be a good idea to check it one more time instead of using a software one time and then sent it to the recruiter.
People with a higher education think differently since they want a specific job
with a good monthly salary and the motivation to work there. 
Above all these software solutions can provide help if quality is not very important factor,
and that will be the pros. For the cons that will be less efficient for people with a high education.

\subsubsection{Streamlining}
If people are applying for a apprenticeship, job or traineeship, 
then it can be a good idea to streamline the CV in the future, because recruiters don't like non relevant information.
As outlined in background(Optimize ones interview rate), 475-600 words seem to be the optimal length for a CV.
Any more then words could lead to non-relevant information and not passing the 7 second CV test.
It would be better to sort out only the most relevant information 
like the education, work experience, to create a to the point CV. 
these scenarios with a eight-page cv can be immediately turn down by the recruiter or 
it could be sorted out of the qualified.
That's why it can be a good idea to keep it short, 
since it is the most effective way of not letting the recruiter to read a long time.

Try to see it from the recruiter's hiring perspective. Every recruiter have different expectations, 
and that is where the streamlining begins. Streamlining has multiple pros as a software solution,
for example it does only need a few user input, so it will automatically create a cv with a sorting method. 
It's better to use valuable space for something than a mass of non relevant texts.
Streamlining has a good opportunity to make a lot of CVs that will help people use less time than making it manually,
and it would actually be beneficial for both parties since they both have to use less time.
Promotes skills and experience can help a lot, but writing can take time, and streamlining can help to do it quicker 
and print out at last a pdf file automatically, so it is ready to use.
There is a chance that all of an applicant's experience does not necessarily contribute to the CV, 
because the recruiter is not interested the positions he/she had before, they want to know what the person have achieved
or the contributions the applicant have made. 

\subsubsection{Structuring/producing a CV}
Structuring and producing a CV is an essential factor, this is so the content is suitable for any job position. 
Structure is what feasible to the hiring personnel and ATS scanner. 
Structuring a CV , this is partly due to having the CV to at least getting through the qualifying test also known as (ATS).
Both keyword matching and formatting are essential to pass this test.
Structuring is not only the structure of the paragraph itself, it is also the structure of the content in the CV.
Structuring a CV is to state relevant information in a structured and practical layout. 
This also includes how to produce the file of the CV, the most popular being as a pdf file.
%https://zety.com/blog/cv-format#structure
Some information can be structured differently for prioritizing the text, which can be about contact information and free text.
Free text will have work experience included, since this particular reason can be sorted out, but only some of it. 
If a person have to work as an Chief Officer for example software like CEO, COO or CCO, 
then it can be a good idea to remove a supermarket career in the CV because of it's not relevant for applying the job. 
Free text is like the subcategory for writing a resume,
and provides an overview of qualifications and skills. The resume can be known an a sketch for the rest of the CV. 
It will be very good at producing a CV, if there is keywords with relevant information, 
and write down that you can be a good candidate for the job.
Professional achievements is at good way to make the CV more interesting, so write them down in the resume.

{
All CV´s will are required certain aspects to first; be considered a CV and secondly to convey one is suited for a job to the hiring personnel.
These aspects are outlined in background(Required and situational content of a CV).
The question of how to format the content of the CV changes with each application, as each firm requires different relevant content.
As formatting is the first thing hiring personnel will see and judge, it is essential that a CV´s content is structured is excellent and makes the content easily accessible.


}
While making a professional CV it can be frustrating to do a specific design on a it for every job, 
since every company has their own taste of CV, but most of the time there are standard types of CVs. 
Those people who are going to apply either a popular company, big company or
in the top public sector have to write a overwhelming good cv and of course good formatting.
Even though low educated people doesn't need to have a lot of relevant information they should still have a good formatting, 
every recruiter need to understand what the applicant have written, that is why it must be readable text, good form 
and good layout that can give a smooth transition for the next paragraph. 
So a software solution to format a CV can be useful for design the content, contact information and a short resume,
and there are also other factors in place to an automatically format generator. Again they are almost no user input to be needed, 
so the amount of time using to format a CV is pretty short. 
LaTeX is a good way to format a CV and making the pdf-file. The system has a standard way of writing, and in this software solution
the applicant doesn't need to do anything while processing through the file. The software solution will take a CV, scan through it and
then format it in a LaTeX file, so the inch margin, line spacing, font, header format and titles 
will be like a pdf with LaTeX. 
%https://zety.com/blog/cv-format#structure
The good way to structure with LaTeX could be to incorporate the important text in the upper side of the CV, 
since all kind of latin language are reading from the top left corner to the low right corner, therefore it can be better
for both recruiters and applicants to have this format.

\subsection{Narrowing of the analysis}
To summarize, we have narrowed down the key aspects in our analysis to the
following:
\begin{itemize}
  \item The solution has to be a software program.
  \item The greatest societal problem is how we can decrease the unemployment rate, 
  therefore the solution is gonna be one that primarily helps the unemployed, 
  even if it may help the employed as well.
  \item Due to the direction of reading, culture differences, 
  and therefore structural differences in applications, the solution is gonna focus
  on how to help those that are Danish and English speaking.
  \item The solution has to be a general one, such that it works in all major industries.
  \item The solutions primary function is aiding in the process of getting a job interview.
\end{itemize}
