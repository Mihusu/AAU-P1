\section{Analysis}\label{sec:analysis}

\subsection{Target group}
If the process of obtaining a job by CV could be optimized to increase the likelihood of obtaining a job, it would benefit most if not all applicants.
As outlined in background(Obtaining a job and the process of applying for a job) five stages exist to obtaining a job by means of CV.
If the process of obtaining a job by CV is to be optimized, one of these stages must be optimized. 
However, optimizing one stage may have a negative effect on another stage.
Therefore, attempting to optimizing the entire process for all applicants is a futile endeavour.
Furthermore, a process optimized for the majority of applicants may fail to optimize anything as the majority of applicants is an extensive target group.
As such, it would be an advantage to choose a target group that can clearly benefit from certain optimized stages.
It also stand to reason, that choosing a target group that benefits most from obtaining a job would be an advantage.
Using what has been discussed in background, it is possible to choose a target group that would benefit greatly from an optimization of the process of obtaining a job. \\

As outlined in Background(Obtaining a job and the process of applying for a job), four types of job searchers exist; unemployed searchers, active searchers, employed tiptoers and employed passive.
Taking the four types as the four possible target groups, it is possible to discern which type would benefit most from being the target group.
As shown in figure(hiringpeople.png) the unemployed searchers and active searchers have a significantly higher likelihood of obtaining a job by means of CV.
Therefore, they would benefit more from a optimized process of obtaining a job then the employed job searchers.
As such, the unemployed searchers and active searchers are the two most relevant target groups for a optimized process.
When analyzing which of these two target groups would benefit more, it is essential to define what would constitute "benefit more".
Two factors have the potential to benefit, the individuals chances of obtaining a job and the countries economy which the individual resides.
Benefiting more would mean either a higher chance og obtaining a job or a positive effect on a countries economy.
As discussed in Background(The effects of unemployment on country and individual), unemployment can have severe negative effects on both economy and the unemployed individual.
Therefore optimizing the process for the unemployed has already established positive effects on both economy and individual.
While active searchers could potentially benefit just as much as unemployed searchers:
A lack of sources supporting the benefits of optimizing the process for specifically active searchers indicates a insignificant benefit.
While an abundance of sources outline the benefits of optimizing the process for unemployed searchers exists, as seen in Background(The effects of unemployment on country and individual).
As such, the unemployed searchers would benefit more from a optimized process then the active searchers.
Therefore, by comparing the four types of job searchers, it can be concluded that the unemployed searchers would be the most justifiable target group for an optimized process of obtaining a job by CV.
Consequently, any optimizations to the process of obtaining a job will be considered with hindsight to the unemployed. \\
 
\subsection{Factors to optimize}
To optimize the process of obtaining a job by CV for the unemployed, an unemployment applicant must have a higher chance of obtaining a job by CV.
However, to optimize the process, it is essential to know what factors have an influence on the likelihood of an applicant obtaining a job.
As these factors are what dictates the unemployed chances, any increased chances of obtaining a job would have to come from optimizing one of the factors.
But what exactly are the factors that determine the chances of an unemployed obtaining a job?
Using what has been discussed in Background, it is possible to identify each factor and analyze how it could be optimized. \\

As outlined in Background(Obtaining a job and the process of applying for a job) the process of obtaining a job has five stages.
If an unemployed applicant were to obtain a job they would have to pass through these five stages.
Different factors will determine the unemployed applicant success in obtaining a job.
It is vital to determine these factors as this will indicate what could potentially increase an unemployment applicants chances of obtaining a job.
Therefore, leading to what a optimized process of obtaining a job by CV could look like for the unemployed.
Using what has been discussed in background and further sources, it is possible to analyze which factors will determine if an unemployed applicants obtains a job. \\

Many factors have an influence on an unemployed applicants chances of obtaining a job.
That does not necessarily mean that all factors must be optimized in order to obtain a job.
However, optimizing the process of obtaining a job will must likely mean optimizing one or several factors.
The most influential factors absolutely necessary to obtaining a job are: \\
\begin{itemize}
  \item  Unemployed applicant´s qualifications
  \item  Time spent unemployed
  \item  Passing the ATS scanner
  \item  Passing the hiring personnel
  \item  Passing the job interview
  \item  Quantity of CV´s 
  \item  Quality of CV´s
  \item  Economy of country
\end{itemize} 
It is essential to understand that factors will overlap and in some cases two factors can be optimized by the same solution.
To understand how one or more of these factors could be optimized, each factor deserves an explanation of its correlation with obtaining a job: \\

An unemployed applicant´s qualifications are essential and as the five stages of obtaining a job purpose is to choose the most qualified applicant, the applicant with relevant qualifications has the natural advantage.
However, relevant qualifications can be a difficult for many unemployment applicants to obtain.
This is due to it becoming harder to to obtain a job as an unemployed as time on, as discussed in Background(process.)
With relevant qualifications it becomes easier to obtain a job, leading to more relevant qualifications, as such success leads to success, this is known as the "job life cycle".\cite{Job_Cycle}
Therefore, a process that increases the chances of obtaining a job will increase the magnitude of qualifications for the unemployed applicant.
However, the job life cycle can also work in reverse.
Meaning that few or no qualifications may lead to not obtaining a job and therefore no new relevant qualifications.
This may lead to a longer period of unemployment, which is the second factor; time spent unemployed. \\

As seen in figure(Unemployment Time.png), the likelihood of obtaining a job declines drastically during short-term unemployment, that is the first 27 weeks of unemployment.
Therefore, the time spent unemployed is a factor that has more influence on an unemployed applicants chances of obtaining a job the longer they are unemployed.
As such, the unemployed applicants qualifications can be optimized through other factors, as it entails obtaining a job.
While the time spent unemployed can be optimized by decreasing the time needed to obtain a job. \\

As discussed in Background(ATS scanner and hiring personnel), to obtain a job an applicant is required to pass the ATS scanner and hiring personnel.
Therefore, passing the ATS scanner and hiring personnel are very non-ignorable factors in an unemployed applicants obtaining a job.
Firstly, as outlined in Background(ATS scanner and hiring personnel), the ATS scanner scans for three underlying elements in an applicants CV.
The ATS scanner scans by searching for pre-defined keywords relevant to the job
Therefore, if a process could incorporate these keywords into a CV, it would increase the likelihood of passing the ATS scanner.
Secondly, the hiring personnel will glance at an applicants CV, spending an average of 7 seconds on each CV.
An applicants CV must be able to leave a good first impression and pass the "7 second test".
As outlined in Background(ATS scanner and hiring personnel), three concepts largely determine the chances of passing the hiring personnel.
As such, if a process could integrate these three concepts into an unemployed applicants CV, it would increase the likelihood of passing the hiring personnel. \\

Passing the job interview is an important factor for the unemployed to obtain a job.
Unlike the relative predictability of the ATS scanner and hiring personnel due to a shared objective of screening applicants CV´s.
The job interview is very unpredictable, possible because it attempts to screen the applicant directly, not through the CV that the applicant applied with.
It also factors in the human element, as the interviewer could make choices based on bias or mood.
Trg claims that "there is little relationship between candidates’ performance in interviews and subsequent on-the-job performance".\cite{Job_Interview}
The job interview is also the final stage in obtaining a job before the job offer, so the hiring personnel will seek any specialized skills that are unique to that job.
As such, the lack of a system other then practicing the large pool of commonly asked interview questions, makes passing the job interview factor hard to optimize.\cite{Job_interview_common_questions}
An unemployed applicant will also naturally improve at conducting job interviews the more job interviews the applicant participates in.
So, optimizing this factor over time can be done by focusing on factors that will net more job interviews and subsequently more job offers. \\

The quality and quantity of CV applications an unemployed individual sends is a important factor in obtaining a job.
Quality and quantity are very loose terms, but are meant to be so, as many overlapping factors and circumstantial aspects have an affect on the quality and quantity of CV´s.
Quality, is the likelihood that a CV will net an unemployed applicant a job.
Therefore, a quality CV is not necessarily the most time consuming CV, it is the CV most likely to pass all five stages of obtaining a job.
Optimizing for a high-quality CV would means an increased chance of each individual CV obtaining a job for the unemployed applicant.
On the other hand, quantity is the amount of CV´s an unemployed applicant will have the potential to apply with.
However, the quality and quantity of CV are inversely correlated.
As such, an increase in quality will lead to a decrease in the quantity of CV´s sent.
Therefore, any optimization to quality or quantity will have to be weighed carefully. \\

As discussed in Background(What is unemployment and how does it materialize), unemployment can materialize by four different types of unemployment.
As unemployment has an inverse correlation with the economy of a country, the state of the economy will influence the likelihood an unemployed obtaining a job.
As such, in a robust economy the chance of obtaining a job will increase, while in a fragile economy the chances of obtaining a job will decrease.
It is improbable that this factor can in any way be directly optimized.
However, a optimized process of obtaining a job could lead to less unemployment in a an economy over time.

\subsection{Functions of a software solution}
A software solution can be many things where it can go from keyword matching to print a pdf file for a CV.
More realistic CV can take it to a balanced level that are not very ambitious, 
so, the recruiter are more interested to have you. Every time the CV needs to be edited
then the applicant need to use the time to be quick and efficient, and afterwords it can be reuse.
Simply put software solutions will make it easier for applicant to apply for a job instead of using a long time to edit and write. \\

Earlier in the text there were written some information 
that quantity and quality of a software solution will be the most optimal for a situational content,
but it depends on which people will be using it, because low educated have a more tendency to apply
in a wide area of jobs without thinking too much about the category. 
Even though quantity and quality contents would be great for the applicant, 
there can still be doubts for the implemented software, since a high educated CV can have many standardized format and information,
and therefore, it can be a good idea to check it one more time instead of using a software one time and then sent it to the recruiter. \\

People with a higher education think differently since they want a specific job
with a good monthly salary and the motivation to work there. 
Above all these software solutions can provide help if quality is not very important factor,
and that will be the pros. For the cons that will be less efficient for people with a high education.

\subsubsection{Streamlining}
If people are applying for a apprenticeship, job or traineeship, 
then it can be a good idea to streamline the CV in the future, because recruiters don't like non relevant information.
As outlined in background(Optimize ones interview rate), 475-600 words seem to be the optimal length for a CV.
Any more then words could lead to non-relevant information and not passing the 7 second CV test. \\

It would be better to sort out only the most relevant information 
like the education, work experience, to create a to the point CV. 
these scenarios with a eight-page cv can be immediately turn down by the recruiter or 
it could be sorted out of the qualified.
That's why it can be a good idea to keep it short, 
since it is the most effective way of not letting the recruiter to read a long time. \\

Try to see it from the recruiter's hiring perspective. Every recruiter have different expectations, 
and that is where the streamlining begins. Streamlining has multiple pros as a software solution,
for example it does only need a few user input, so it will automatically create a cv with a sorting method. 
It's better to use valuable space for something than a mass of non relevant texts. \\

Streamlining has a good opportunity to make a lot of CVs that will help people use less time than making it manually,
and it would actually be beneficial for both parties since they both have to use less time.
Promotes skills and experience can help a lot, but writing can take time, and streamlining can help to do it quicker 
and print out at last a pdf file automatically, so it is ready to use.
There is a chance that all of an applicant's experience does not necessarily contribute to the CV, 
because the recruiter is not interested the positions he/she had before, they want to know what the person have achieved
or the contributions the applicant have made. 

\subsubsection{Structuring and producing a CV}
For a CV to be effective it must both organize its content in an adequate manner and be accessible to the hiring personnel when sent.
It is therefore essential to analyze both the structure and production of a CV. \\

Structure is crucial, as it organizes the content of a CV so it is passable to the hiring personnel.
As outlined in background(Required and situational content of a CV), a CV is most effective by integrating the right quantity and quality of situational content.
In essence, a majority of applicants CV's viewed by the hiring personnel follow the "7 second test".
As structure is the first thing hiring personnel will see and judge when glancing over a CV, it is essential that a CV´s structure conveys competence.
Therefore having an accessible structure that invites further examination and thereafter a context that fits the overall structure;
is optimal and should be the intention of any effective CV program. \\

As stated before, the structure of a CV has to fluctuate from application to application to be effective.
As outlined in background(Structures of the CV), different CV´s should be used for different applicants.
The digital CV program´s main beneficiary group are the unemployed therefore we must choose a structure that on average suits them best. \\

The production of a CV is meaningful as it refers to the type of file the CV will be sent as.
It is essential that a CV, once structured correctly is accessible and practical for both the hiring personnel and ATS scanner.
The two most popular way methods of production are the PDF file and Word document. \\

If the job specifies a preference for one form or the other, the applicant should follow the advice.
However if not specified in most cases the PDF file is superior.
This is because a wide range of programs can be used to open and view PDF files and the file will look the same no matter which computer is used to open it.
The PDF file also prevents hiring personnel from making changes to the CV. 
The only advantage of the Word document being that it tends to takes up less space in a computers storage then the average PDF file.
Therefore the optimal production of a CV is in most cases the PDF file. \\

So a software solution to format a CV can be useful for design the content, contact information and a short resume,
and there are also other factors in place to an automatically format generator. Again they are almost no user input to be needed, 
so the amount of time using to format a CV is pretty short. 
the applicant doesn't need to do anything while processing through the file. The software solution will take a CV, scan through it and
then format it in a LaTeX file, so the inch margin, line spacing, font, header format and titles 
will be like a pdf with LaTeX.\\

The good way to structure with LaTeX could be to incorporate the important text in the upper side of the CV, 
since all kind of latin language are reading from the top left corner to the low right corner, therefore it can be better
for both recruiters and applicants to have this format\cite{Pdf_vs_word}.\\

\subsection{Keywords}
As stated previously 90 percent of companies use a ATS scanner that scans for keywords relevant to the position.
If a CV does not contain enough relevant keywords it is rejected. 
Although estimations vary, an average of 75 percent of applications get rejected by the ATS scanner.
So, what does a ATS scanner actually scan after in a CV and what does it do with what is scans? \\

As outlined in background "" the ATS scanner scans for two things in your CV:
Firstly, the ATS scans you whole CV for certain keywords, chosen by the firm, as being relevant for position.
Secondly, the ATS scans specifically for your work experience and education, 
as keywords pertaining to these are chosen by the firm as being relevant. 
The more relevant keywords a CV has the more points the ATS scanner will give 
and the higher ranked the CV will be listed when viewed by hiring personnel. \\

Most positions will have a threshold of points, if a CV does not meet this threshold, 
it will be rejected and will not proceed to the hiring personnel.
In other words it is the minimum to pass the ATS scanner, however the more relevant keywords a CV contains for each position, 
the more effective the CV process will become. \\

The objective of a more effective CV process is therefore to, at the minimum, to pass the two scans of the ATS.
It is important to analyze what a solution to both scans would have to entail. \\
For the scan of the entire CV for keywords, it is important to note that the firm choses the keywords which will be relevant.
The most effective method of passing this scan would be to directly implement the same relevant keywords that the ATS scanner is scanning for.
However, the applicants and CV process will have only the position and the job posting to guess and thereafter implement the relevant keywords.
If the relevant keywords are found and implemented into the applicants CV, this would in the majority of cases pass the ATS scanner. \\

For the scan on work experience and education, the objective of the firm is to see if the applicant has the qualifications.
Again it is vital to note that the firm decides what the ATS scanner will find to be relevant work experience and education.
However, most job postings will have a section listing the minimum qualifications for the position, this gives a clear minimum for what work experience and education must contain.
The first step is to make absolute the ATS scanner is able to easily identify and scan the applicants qualifications. \\

A clear differentiation of education and work experience from other sections of the CV can therefore be beneficial.
To make the substance of work experience and education pass the ATS scanner the applicant must at the minimum have the qualifications for the position, most likely seen in the job posting.
If the qualifications of the applicant are structured correctly it would in the majority of cases pass the ATS scanner.
As analysed, what solutions to pass both parts of the ATS scanner would have to entail, with the potential to make a more effective CV process. \\


