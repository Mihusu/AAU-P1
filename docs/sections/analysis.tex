\section{Analysis}\label{sec:analysis}




\subsection{Solutions}
One of the greatest problems with the process of job hunting, is just how inpersonal
the reqruitment process is. As stated earlier, one of the most inpersonl aspects
of finding a job is that using the wrong keywords can very
quickly land your CV in the spam folder.
Even if your CV gets through, it might be rejected and thrown away because of
minut errors, since the reqruiter can't judge you as a person, through the CV.
This fact can easily be turned around to your advantage, as you can optimize the
job hunting process in three steps:
\begin{itemize}
  \item Get through the keyword detection algorithm
  \item Optimize the quality of the CV
  \item Send a large quantity of CVs
\end{itemize}

It is easily seen, that the the stiffer the competition is, the more one 
needs to focus on the first aspect. This is evident by the fact, that a 
recruiter only wants a few of the most relevant applications.
The more specialized one's role is, the more one needs to focus on the second
aspect. This is evident by the fact, that not many will be eligable to qualify
for an advanced specialized role.
The more generel the job is, the more you need to focus on the third
aspect. This is clearly seen, as the more generel the job is, the more can 
qualify to fill that role. This makes the job-hunting process more about luck.
Luckily is mearly a function of probability, which can be optimized using
quantity.

This makes a software solution an optimal way, of increasing the job hunting
process, since all three of these aspects can drastically be improved and automated
using specialized software. 
The first step can easily be fitered through, to translate all relevant, but
misspelled or incorrectly worded, keywords into their correct form. Such
as the word "cocktail expert" into "bartender" using software.

The second point can be fixed through many thousands of data points, using
artificial intelligens, to aid how to formulat oneself properly. 

The third step is properly what software is best at: Repetitive and iterative
tasks.
One could automate the filtering of irrelevant parts of a CV out, and then send
many thousand of pseudo specialized CVs out without requiring much work from the
user.

Since we want to reduce the general unemployment rate, the solution is one that needs to 
be as general as possible. Therefor fixing 'step one' is alpha omega.
From there, we could create different tools, to help one create the most relevant
CV for the situation the applicant is in.

The quality aspect (step two) is more important for more specialized roles, 
but since the intent is to help as many people as possible then the solution
must be a aiding program, and not a writing program.
This is the case, since specialized roles are far and few inbetween,
as such, 'step two', is not the biggest issue, for most people. 

'Step three' is also of a greater importance, since being able to create a
greater quantity of applications, drastically increasses one's chance of
employment. 
As such having a software solution be able to automate all the most boring
bits of writing applications is incredible useful.

Some people wish to spend time on writing applications, but either have
dissabilities or other issues making it hard to send an appropiate amount
of applications out. Having a program to automate most of the process, enhances
the challenged peoples' ability to acquire a job.

It could be argued, that having everyone use aiding software to enhance ones
quantity of applications would eventually even out the playing field, making
it such that we have the same problem as before. 
But we do not think this is the case, since this kind of software only seeks
to eliminate the many firewalls reqruiters use to filter people out, be it 
qualified applicants or not.
This, inturn makes the playing field a field, where one is not determined by
simply mistakes, but hopefully more on one's merit. Since reqruiters will have to 
individually sort through bad applicants, instead of having a program that sorts
all the applicants with simply mistakes out. In the worst-case scenario, more
people would have enough energy to send an appropiate quanity of applications
out, and recruiters will have more people to choose from, in turn increasing
equality of opportunity.



