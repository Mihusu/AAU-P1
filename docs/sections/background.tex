\section{Background}\label{ch:background}

\subsection{What is unemployment and how does it materialize}
Every country is built on their population producing something of value. 
In an utopia, all citizens of each country would produce value enough to cover themselves and a little more for the structures of there country.
However, in reality the value of each citizens production value fluctuates. 
Some citizens are hyper-active and produce immense value for their country.
While in every corner of the world, members who produce little to nothing are also found. 
In the modern economic infrastructure the citizens known the "unemployed" make up a portion of those without value.
However, they have the potential to produce and support there respective countries.
But what exactly is unemployment and how does it materialize? \\

Unemployment occurs when a person who is available and actively searching for employment is unable to find work. \cite{Guide_to_unemployment} 
At first glance it is difficult to tell what types of people unemployment encompasses.
Although we can see that unemployment requires two factors, for the person to be available for work and be actively searching for work.
Firstly, being available means to not be preoccupied part-time or full-time with another occupation.
Therefore unemployment does not for example include children, the retired, full-time students, part-time workers, disabled or those on maternity leave.
Secondly, to be actively searching entails that the person has actively looked for work in the prior four weeks. \cite{US_unemployment_statistics_definition} 
So, those individuals who are jobless but not actively searching for work are not unemployed.
Now, it has been discussed what unemployment is, the next step it to examine why it materializes. \\

Unemployment is found in every economy and society, but how does it materialize?
Unemployment could materialize for many different reasons, to understand these reasons it is useful to divide unemployment into four different types.
Thereafter discuss what each type encompasses and how the unemployed could be affected.
Four types of unemployment: \\
\begin{itemize}
   \item  Structural unemployment
   \item  Cyclical unemployment
   \item  Frictional unemployment
   \item  Seasonal unemployment
   \end{itemize} \cite{Four_types_of_unemployment} 
Structural unemployment occurs if there is a mismatch between offered and demanded skills.
This could be a lack of demand for workers of a certain skill set, or an excess supply of a job with a lack of workers with the matching skill set.
For the unemployed, it often required to learn a new skill sets or further educate themselves to gain a job.
Cyclical unemployment arises if there is a downturn in the economy and no jobs are available.
This is the biggest cause of unemployment and can have significant consequences on unemployment globally. \cite{Understanding_four_types_of_unemployment}
It can develop when there is a reduction in the demand for a firms products of services and the firm therefore has no need for high production, cutting back on there workforce. 
Frictional unemployment refers to workers who are in transition between jobs. 
It is not entirely a bad thing as often it is caused by a worker finding a job more suitable for their skills.
This also involves those workers who recently left or were fired and are actively searching for a job.
Seasonal unemployment occurs when the demand for workers varies throughout the year.
This type of unemployment often refers to climate dependant economic sectors, such as agriculture or tourism.
It is fairly predictable in most cases and often requires workers to find another occupation for the rest of the year. 
As closure, four potential ways for unemployment to materialize have been discussed.
Unemployment can potentially materialize from almost any economical, psychological, cultural, seasonal or institutional reason. 
A more economical method would conclude that unemployment could materialize "from both demand side, or employer, and the supply side, or the worker." \cite{Economical_theory_behind_unemployment}
Now, it has been discussed how unemployment can materialize through the four types of unemployment. \cite{Guide_to_unemployment}   \\

\subsubsection{The effects of unemployment on country and individual}
Unemployment is not a consistent state, as discussed in background(What is unemployment and how does it materialize), unemployment can materialize due to a variety of conditions.
However, the discussion of what unemployment is and how it materializes, leaves out the individuals and country effected.
Consequently, it is meaningful to further discuss unemployment and how it effects both country and the individual.
However, to make any conclusions on the effects of unemployment, it must refer to statistical measurements of unemployment, country and individual. 
Therefore, the methods used to measure unemployment, country and individual will also be discussed.
So, how does unemployment effect both country and individual, and how do we measure it?

To gain an understanding of unemployment´s effect on both country and individual it is vital to first define what the country and individual encompasses.
As both country and the individual are loose and imprecise terms it is helpful to narrow the discussion down to a couple countries and a group of relevant individuals.
Unemployment differs wildly from country to country, so to choose two countries similar enough to to make definitive findings on unemployment is productive.
Furthermore, it is valuable to choose countries which are relatively stable and reliable in there unemployment statistics so research and findings are constructive.
The two countries that are optimal as candidates are The United States of America and the United Kingdom.
The US and UK are both relatively stable and have reliable statistics within comparably similar socio-economic structured countries. \cite{Economic_similarities_US_UK}
Additionally, "effected" is quite a sweeping term, so the discussion can be furthermore narrowed down to how unemployment effects the economics of the US and the UK.
Similarly as discussing all countries, discussing the effect on all individuals in the USA and UK would be too general and speculative.
Therefore, it will be most productive to discuss those most directly effected by unemployment, the unemployed.
As unemployment can have almost any imaginable effect on the unemployed, it is helpful to focus exclusively on how unemployment directly effects the unemployed´s chances of obtaining a job.
Thus, the discussion will be narrowed down to; how does unemployment effect the US and UK economies and there unemployed prospects of obtaining a job?

Firstly, how does unemployment effect the US and UK economies?
The first obstacle to answering the proposition is to find how to represent unemployment and the economies of the two countries in statistics.
A good statistical representation of unemployment is the number of unemployed, represented by the unemployment rate of a country.
While the economies will be represented as the growth rate of the GNP(gross national product) and inflation of an economy.
Therefore, depending on the correlation between unemployment rate and the growth rate and inflation in a country, it is possible to see the effects of unemployment on the US and UK economy.
First discussing the unemployment rate and growth rate of the GNP of the US then UK.
By inserting the unemployment rate and growth rate into the same graph a correlation might be seen:

\incfigure{figures/United States_Growth_Rate(GNP)_and_Unemployment_Rate}{fig:rate_US}{Unemployment and Growth rate \cite{US_Unemployment}\cite{US_Growth_Rate_GNP}}

Here we see a somewhat clear correlation between growth rate of the GNP and unemployment rate of the US economy.
A see a inverse correlation between the two, meaning that when one decreases the other increases and vice versa.
If we look at the effect of the unemployment rate on the growth rate in the UK it should mirror this conclusion.

\incfigure{figures/United_Kingdom_Growth_Rate_(GNP)_and_Unemployment_Rate}{fig:rate_UK}{Unemployment and Growth rate \cite{UK_Unemployment}\cite{UK_Growth_Rate_GNP}}

As shown, a clear inverse correlation between the growth rate of the GNP and unemployment rate can again be concluded in the UK.
This was expected and supported by the economic theory; Okun´s law.
Okun´s law states that the unemployment rate and GNP of a country have an inverse correlation.
Therefore meaning that the unemployment rate would also hae and inverse effect with the growth rate of the GNP. https://www.investopedia.com/terms/o/okunslaw.asp

\subsubsection{The numbers behind the process of getting a job}
Both after the aftermath of the 2008 recession, and now in a more stable
economy, all of the vacant job positions needs to be
filled. But just how did people end up acquiring their current job?
Lou Adler tried answering just this: He conducted an online survey
on LinkedIn based on 3000 answers, where in most of these answers
came from those actually hiring.

The results are outlined in \vref{fig:hiring}.
\incfigure{figures/hiringpeople}{fig:hiring}{How people get jobs 2015 and 2016 \cite{Networking}}

Here we can see, that active candidates only represent 5-20 percent of the
entire job market. Around 15-20 percent are only tiptoeing around the idea
of getting a new job, while the rest are passive candidates (meaning people who are
satisfied in their current position.).

Based on these graphs, we can see that at the very minicaptionmum, 42 percent of people are
hired based on networking, and that is only if you already have a job.
Whilst this percentage only goes up, the opportunity for a job seeking person
to get hired based on an application goes down. This reflects the
most effective way to get hired is through internal applications or networking.
On the flip side, it also nicely illustrates just how stiff competition there actually is,
if your only way of getting hired is through sending applications.

When you send out an application, there is an 8.3 percent probability, that
they will actually invite you to a job interview. Furthermore it takes around
10-15 interviews, before one gets a job offer. Obviously it varies depending
on educational background, job type and many other factors, but this is the average.
Some quick math ( (((100/8,3)*10)+((100/8,3)*15))/2 = 150..) tells us, that it will
take an average of 150 ish applications before one gets a job offer.\cite{HR-sales}

All these applications add up, before one can reap the reward. According to a
study conducted surveying 2000 Americans, by recruitment agency Randstad US
discovering the “art of the job hunt”, it takes an average of five
months from when the job search begins, until one actually lands
the job. \cite{5_month_for_a_job}

To add insult to injury, after all the hard work of creating an application, it can
take quite some time before the hiring managers actually respond, that is if they ever
bother answering your application to begin with.
It takes around 3 days between they receive the application, before they answer.
This is the case for the most in demand roles in society, for the less in demand roles
such as writers, nurses and unskilled labour, it can be anywhere from 10 to over 30 days.\cite{HR-sales}
On average one can expect to hear back from employers within a week 41 percent
of the time. Within a couple of weeks 85 percent of the time.\cite{Hear_back_your_job}

Below some of the more popular jobs are illustrated as a function of interview
rate on the left and response delay on the left:
\incfigure{figures/interviewratexdelay}{fig:delay}{Caption?\cite{HR-sales}}
The interview rate is further supported from a danish online survey, that concluded
that 65 percent of people get an interview within the first 15 applications and
82.5 percent of people get an interview within the first 30 applications.\cite{Amount_of_applications}
\clearpage

\subsubsection{Optimize ones interview rate}
There are many factors to consider, if one wishes to optimize ones
chances of getting an interview, one of the more empirical proven ones
is what time and day one sends their application.
To get the highest chances, you have to apply between early Tuesday morning
and Thursday before noon using the employers local time. Monday is even better,
increasing your chances by 46 percent in regard to the average.
If one should apply on another day, the most important factor is that
it's done before 10AM, since the interview chances drops below 5 percent for
the majority of late evening applications.\cite{Best_time_and_date}

One thing that is crucial, is how long ones resume is:
Having between 475-600 words is the optimal length, where the maxima is at
ca. 535 words. The optimal word count is illustrated in \vref{fig:length}.
\incfigure{figures/longresumegraph}{fig:length}{Optimal length of a CV\cite{Job_Application_for_science}}

Perhaps one of the most influential condition of wether you get an interview,
is how fast you are at applying:
Based on 30000 data points from the company Speedrecruiters, you need
to apply within the first 14 days to have a practical chance of getting an
interview. It is such that 50 percent of the people who got an interview
for the job applied within the first week, and 75 percent of those who
got an interview applied within the first 14 days, whilst the chances of
getting an interview thereafter dwindles exponentially.\cite{Best_time_and_dateV2}

Other factors that influence the interview rate are as following:
\begin{enumerate}
\item Being a woman increases the chances by 48 percent.
\item Being older, but no older than 35, increases the interview rate by 25 percent.
\item Having more than one degree, increases ones chances by 22 percent.
\item Adding industry buzzwords increases your chances by 29 percent.
   e.g. If you are a software developer, then add buzzwords such as machine learning,
   artificial intelligence etc.
\item Demonstrating earlier job results using numbers increases chances by 40 percent.
   e.g. "Increased profits  by 20 percent from Q3 to Q4"
\item Listing achievements, where you weren't in charge, but only a helping hand
 decreases your chances by 50 percent
   e.g. "Helped management organize financial reports" instead of "Organized financial reports"
\item Using leadership affiliated buzzwords increases your chances by 51 percent
\item Not using personal pronouns in the employment section increases your
chances by 55 percent.
\item Including a key skills section and buzzwords of the key skills increases your
 chances by 59 percent
\item Start ones sentences with distinct action verbs, increases ones chances by 140 percent.
   e.g. Do "Developed a mainframe architecture that dramatically increased efficiency"
   instead of "After surveying people, the mainframe architecture that increases efficiency was
   developed by me."\cite{Science_job}
\end{enumerate}


%how long does it take to actually get a job

%how do these people then end up getting the job

%for the people sending applications, how many did they sen? ++ details

%best time to send applications, and why it can benefit people who
%don't have time there ++ howl ong should you wait with a followup

\subsection{Different expectations in a company}
Companies have different expectations for a CV, because their specific requirements can vary from old job to new.
Most of the time the companies expectations can be related to work experience and education,
and a CV could either be long or short depends on which company the person is writing to. Some people can write a long CV,
but it isn't necessarily a good CV, and people can write a short one but it isn't good enough. \\

Both of these statements could have some information that are not relevant to the company's requirement. Still there are some other
factors that can be included, and some companies would love to know what the person did in that particular year.
In that particular situation would be different from company to company, since those people who are working with humanities
can have human related criteria for getting a job in this area. The same goes for IT where they have more work with a computer
than any other people, because these jobs immerse themselves everyday with it. According to Computerworld it-jobbank,
they have actually examined a total of 6700 job posting in the year of 2015 and 2019 where the company was sorting all categories
that are related to IT, and they have published the top 10. It is shown between those years that the job-advertisements
have more of a technical, structured and professional knowledgeable side than before.\cite{10_personal_skills} \\

That is very common, since will take good technical skills to do a very good job with the programming,
and structured can be a very important factor to have a good overview of a program,
and even if a new person with almost zero experience had to look at that,
it would still be possible to read trough the comments and understand it's functions. Below is the full result
of the 6700 job-advertisements that has been thoroughly examinant. The ranks are from top to bottom.

\newpage
Top 10 in 2019
\begin{enumerate}

\item Technical
\item Structured
\item Professional knowledge
\item Strong
\item Dynamic
\item Analytical
\item Responsible
\item Outgoing
\item Curious
\item Professional
\end{enumerate}

Top 10 in 2015
\begin{enumerate}

\item Structured
\item Technical
\item Dynamic
\item Strong
\item Informal
\item Responsible
\item Professional
\item Outgoing
\item Analytical
\item Committed
\end{enumerate} \clearpage

Also It-jobbank have made a Survey and it shows that the most popular phrases
that has been included in a CV can give more points to get a job in the IT companies.
\begin{itemize}
\item Good for creating an overview and structure
\item Good for creating dialogue
\item Good at articulating you in writing and orally
\item Good at prioritizing your tasks
\item Good at seeing connections
\end{itemize}

Not very popular to write:
\begin{itemize}
\item Good to collaborate and share your experiences
\item Good for keeping a cool head
\item Good at sharing your knowledge
\item Good at innovating, challenging and finding untraditional solutions
\item Good at uncovering and understanding customer needs\cite{10_personal_skills}
\end{itemize}

Contents and the document setup can often be different from company to another,
there is one in particular and they are a comparative effectiveness research called
Patient-Centered Outcomes Research Institute (PCORI).
This organization have a purpose to fund research, so they can afterwords help patients and in the end try to make them better
informed to look at their health that they face every day.\cite{About_PCori}
They have a list to guide applicants who wants to work in the American Medical Association (AMA),
and this is only one of the examples of a requirement for a company.
In the situation of the contents and the document setup, there can be often of those different kind of setups,
there is one in particular and they are a comparative effectiveness research, and it's called
Patient-Centered Outcomes Research Institute (PCORI).
It's to determine which work best for which patients and which pose the greatest benefits and harms. % Edit this
They have a list to guide applicants who are want specifically work here:

Header: Include the Principal Investigator’s (PI’s) full name in the top left corner of the page header on every page.
Margins: Use at least half-inch margins. The header may fall within the top margin, but the body text should not begin closer than one half-inch from the edge of the page.
Font: Use size 11 Calibri for the main body of the text. Figures, tables and captions may be size 8 font.
Page Numbering: Each page must be numbered consecutively for each PDF upload. Each section of an uploaded document must begin with page 1.
Spacing: Use single spacing.
Document Format: Upload all attachments in PDF format.\cite{CV_for_PCori}
As for a general CV there are a standard for almost all the companies and that is:
\newpage
\begin{enumerate}

   \item Contact Information
   \item Personal Statement
   \item Professional Experience
   \item Academic History
   \item Key Skills and Qualifications
   \item Industry Awards
   \item Professional Certifications
   \item Publications
   \item Professional Affiliations
   \item Conferences Attended
   \item Additional Training\cite{Format_for_CV}
   \end{enumerate}

Although the applicants can write a good CV, and some of the words were good,
but there is a chance where keywords come in as a very important factor. In the digital world where often people have to send applications online, and over 90 %%
of all resumes and relevant information are being screened through an "Applicant Tracking System" (ATS). ATS is a scanning software system
that is designed to scan a resume for "work experience, skills, education, and other relevant information."\cite{ATS}
If it determines the resume is a good match for the position, it gets sent forward to the hiring manager.
Every time the ATS will do test, and it will determine if the test is a passing grade or it will be delete, so it will not even reach out to the hiring manager.
According to Caitlin Proctor it is stated that "Nearly 75"
of resumes are rejected because they’re not correctly formatted or keyword optimized." and there can be a lot of criteria
to get to the job you want.\cite{ATS}
So ind the end, the ATS is checking especially on five different elements when writing a resume, and that can be:

\begin{itemize}
\item Standard formatting
\item Keyword optimization
\item Send as a Word document
\item Spell out abbreviations
\item Include relevant information\cite{ATS}
\end{itemize}
 

\subsection{Required and situational content of a CV}
A CV is “a short account of one’s career of qualifications prepared typically by an applicant for a position”.\cite{Difference_between_resume_and_curriculum_Vitae}
When applying for a position within a firm, some aspects of a CV are required. \\
These requirements can come from the position or firm that the CV is intended or from the definition of the CV itself.
Putting aside the firms or positions requirements, all CVs must include the five following requirements to be defined as a CV:

\begin{itemize}
   \item 1. Contact information
   \item 2. CV objective
   \item 3. Relevant skills
   \item 4. Work experience
   \item 5. Education\cite{Write_a_curriculum_Vitae} \\
\end{itemize}

The substance of each requirement varies from applicant to applicant. However, every CV must include these five requirements to be effective.
A further explanation of each requirement is due:
Contact information is required as the firm at the bare minimum must have some way to contact the person so he or she can be accepted for the position.
CV objective is required as it specifies what and who the CV is intended and without it the CV can fill no purpose.
Relevant skills are required as without it you have no relation to the CV objective.
Work experience and education is required as without it one has no qualifications. \\

As a CV is an account of one’s qualifications it is also possible to leave education and work experience empty if one has none.
However it is then hardly an effective CV.\cite{Difference_between_resume_and_curriculum_Vitae}
Along with the requirements of CV there are also aspects which are situational.
These vary and most likely the firm or position in which one is applying to will lay out these aspects, or it is apparent from the position itself.
If not specified it is difficult to know what to include. To little and one will under-qualified, to much and the CV will be to long.\cite{Job_Application_for_science}, excesses and unorganized.
Therefore, it is essential to distinguish one’s CV with the right quantity and quality of situational content.
Lets examine some of the more typical situational content that could be included in a CV: \\

\begin{itemize}
   \item 1. Professional association
   \item 2. Volunteer experience
   \item 3. Languages
   \item 4. Additional training courses
   \item 5. Publication
   \item 6. Awards/Honors
   \item 7. References\cite{6_sections} \\
\end{itemize}

The effectiveness of a CV can drastically change due to use of situational content.
Therefore it is necessary to further explain each situational content: \\
Professional association is any trade unions, learned societies, regulatory universities and other inter-professional societies.
Many associations have certain prestiges and hold there member to a certain standard of quality.
Therefore a professional association can improve a CV if relevant.\cite{Professional_associations_and_organizations}\cite{Perks_of_professional_organizations}
Volunteer experience is any volunteer work relevant to the position.
Languages is any spoken or written language relevant to the position.
As most firms in our interconnected market interact with some multilingualism, languages can easily increase the quality of a CV.
Additional training courses are any extra courses relevant to the position. \\

Publication are any reports, books or other published materials that could show qualifications for the given position.
Awards and honors are any university or professional awards or honors given that show qualification for thr given position.
References are very situational, as putting references in a CV may make you seem unsure of yourself and in need of validation from others to show qualifications.
However, if a CV has the right references it can assure employers of your qualifications and past experience.\\

\subsection{Structures of the CV}
Writing an effective isn´s just about including the right content. 
Its also about how you present that information.
Every applicant must have a structure that best presents there CV.
It is therefore relevant to discuss the different structures available to the applicant. 
In theory, an applicant could structure there content a million different ways.
To narrow it down we will not discuss the three most common structures and the pros and cons of each of them.

The first structure is the Chronological CV structure: \\
\begin{itemize}
   \item  Contact information
   \item  CV objective
   \item  Work experience
   \item  Education
   \item  Relevant skills
   \end{itemize}
The Chronological CV structure is the most common structure and gives an easy overview over the applicants content.
It is adaptable and features all required content mentioned in (Required and situational content of a CV).
With the added section "relevant skills" where the applicant can include there situational content to standout.
Its traditional structure is simple for both hiring personnel and the ATS Scanner to view.
The structure is best suited for those who are confident in there CV content, especially there work experience and education.
The structure can have a difficult standing out as it is the most common and should not be used by those who have big gaps in employment.\\

The second structure is the Functional CV structure:
\begin{itemize}
   \item  Contact information
   \item  CV objective
   \item  Relevant skills
   \item  Situational content
   \item  Work experience
   \item  Education\\ 
   \end{itemize}

The Functional CV structure focuses on relevant skills and situational content over work experience.
This structure is favoured by those applicants with large gaps in there unemployment history, or in the middle of a career change.
As the structure emphasizes the applicants relevant skills and situational content it is flexible as the applicant has a chance to present himself on his own terms.
This structure is highly dependant on the applicants relevant skills and how they present there significance to the job they are applying.
The structure is highly flexible and can either be a nightmare for both the hiring personnel and ATS scanner, or if structured correctly with correct usage of keywords be highly effective.\\

The third structure is the Combination CV structure:
\begin{itemize}
   \item  Contact information
   \item  CV objective
   \item  Relevant skills
   \item  Work experience
   \item  Situational content
   \item  Education
   \end{itemize}
The Combination CV structure is as stated, a combination of the Chronological and Functional CV structures. 
It enables the applicant with an otherwise mediocre content to combine the two and compensate a weakness with another CV structure.
This is a highly flexible structure and lets an applicant who knows how to present oneself to standout.
It is not a structure that an applicant with either overwhelming relevant content or no relevant content to use.
As the structure does not allow for a high emphasize on one section without an obvious lack in the others.
   \cite{Resume_structure}
   \cite{Tips_for_best_format}
   \cite{8_Best_cv_format}
\clearpage
