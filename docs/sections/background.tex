\section{Background}\label{ch:background}

\subsection{What is unemployment and how does it materialize}
Every country is built on their population producing something of value. 
In an utopia, all citizens of each country would produce value enough to cover themselves and a little more for the structures of there country.
However, in reality the value of each citizens production value fluctuates. 
Some citizens are hyper-active and produce immense value for their country.
While in every corner of the world, members who produce little to nothing are also found. 
In the modern economic infrastructure the citizens known the "unemployed" make up a portion of those without value.
However, they have the potential to produce and support there respective countries.
But what exactly is unemployment and how does it materialize? \\

Unemployment occurs when a person who is available and actively searching for employment is unable to find work. \cite{Guide_to_unemployment} 
At first glance it is difficult to tell what types of people unemployment encompasses.
Although we can see that unemployment requires two factors, for the person to be available for work and be actively searching for work.
Firstly, being available means to not be preoccupied part-time or full-time with another occupation.
Therefore unemployment does not for example include children, the retired, full-time students, part-time workers, disabled or those on maternity leave.
Secondly, to be actively searching entails that the person has actively looked for work in the prior four weeks. \cite{US_unemployment_statistics_definition} 
So, those individuals who are jobless but not actively searching for work are not unemployed.
Now, it has been discussed what unemployment is, the next step it to examine why it materializes. \\

Unemployment is found in every economy and society, but how does it materialize?
Unemployment could materialize for many different reasons, to understand these reasons it is useful to divide unemployment into four different types.
Thereafter discuss what each type encompasses and how the unemployed could be affected.
Four types of unemployment: \\
\begin{itemize}
   \item  Structural unemployment
   \item  Cyclical unemployment
   \item  Frictional unemployment
   \item  Seasonal unemployment\cite{Four_types_of_unemployment}
\end{itemize} 
Structural unemployment occurs if there is a mismatch between offered and demanded skills.
This could be a lack of demand for workers of a certain skill set, or an excess supply of a job with a lack of workers with the matching skill set.
For the unemployed, it often required to learn a new skill sets or further educate themselves to gain a job.
Cyclical unemployment arises if there is a downturn in the economy and no jobs are available.
This is the biggest cause of unemployment and can have significant consequences on unemployment globally. \cite{Understanding_four_types_of_unemployment}
It can develop when there is a reduction in the demand for a firms products of services and the firm therefore has no need for high production, cutting back on there workforce. 
Frictional unemployment refers to workers who are in transition between jobs. 
It is not entirely a bad thing as often it is caused by a worker finding a job more suitable for their skills.
This also involves those workers who recently left or were fired and are actively searching for a job.
Seasonal unemployment occurs when the demand for workers varies throughout the year.
This type of unemployment often refers to climate dependant economic sectors, such as agriculture or tourism.
It is fairly predictable in most cases and often requires workers to find another occupation for the rest of the year. 
As closure, four potential ways for unemployment to materialize have been discussed.
Unemployment can potentially materialize from almost any economical, psychological, cultural, seasonal or institutional reason. 
A more economical method would conclude that unemployment could materialize "from both demand side, or employer, and the supply side, or the worker." \cite{Economical_theory_behind_unemployment}
Now, it has been discussed how unemployment can materialize through the four types of unemployment. \cite{Guide_to_unemployment}   \\

\subsubsection{The effects of unemployment on country and individual}
Unemployment is not a consistent state, as discussed in background(What is unemployment and how does it materialize), unemployment can materialize due to a variety of conditions.
However, the discussion of what unemployment is and how it materializes, leaves out the individuals and country effected.
Consequently, it is meaningful to further discuss unemployment and how it effects both country and the individual.
However, to make any conclusions on the effects of unemployment, it must refer to statistical measurements of unemployment, country and individual. 
Therefore, the methods used to measure unemployment, country and individual will also be discussed.
So, how does unemployment effect both country and individual, and how do we measure it? \\

To gain an understanding of unemployment´s effect on both country and individual it is vital to first define what the country and individual encompasses.
As both country and the individual are loose and imprecise terms it is helpful to narrow the discussion down to a couple countries and a group of relevant individuals.
Unemployment differs wildly from country to country, so to choose two countries similar enough to to make definitive findings on unemployment is productive.
Furthermore, it is valuable to choose countries which are relatively stable and reliable in there unemployment statistics so research and findings are constructive.
The two countries that are optimal as candidates are The United States of America and the United Kingdom.
The US and UK are both relatively stable and have reliable statistics within comparably similar socio-economic structured countries. \cite{Economic_similarities_US_UK}
Additionally, "effected" is quite a sweeping term, so the discussion can be furthermore narrowed down to how unemployment effects the economics of the US and the UK.
Similarly as discussing all countries, discussing the effect on all individuals in the USA and UK would be too general and speculative.
Therefore, it will be most productive to discuss those most directly effected by unemployment, the unemployed.
As unemployment can have almost any imaginable effect on the unemployed, it is helpful to focus exclusively on how unemployment directly effects the unemployed´s chances of obtaining a job.
Thus, the discussion will be narrowed down to; how does unemployment effect the US and UK economies and there unemployed prospects of obtaining a job? \\

Firstly, how does unemployment effect the US and UK economies?
The first obstacle to answering the question posed is to find how to best represent unemployment and the economies of the two countries in statistics.
Unemployment is simple, as the clear best statistical representation of unemployment is the percentage of unemployed, represented by the unemployment rate of a country.
How to represent the economy of a country is more difficult, economists generally use GDP and inflation as a overall measurement of an economy.
As the unemployment is a percentage statistic, utilizing GDP growth rate will let us measure the two statistics as percentages.
Inflation has no real direct statistic representing represented for a whole country, therefore wage inflation will be used as the most representative statistic.
It is now possible to compare the unemployment rate with wage inflation and GDP growth rate.
Depending on the correlation between unemployment rate and the GDP growth rate and wage inflation in a country, it is possible to see the effects of unemployment on the US and UK economy.
Discussing the unemployment rate and growth rate of the GDP of the US then UK.
By inserting the unemployment rate and growth rate into the same graph a correlation might be seen: \\

\incfigure{figures/United_States_GDP_Unemployment}{fig:rate_US}{Unemployment and Growth rate \cite{US_Unemployment}\cite{US_Growth_Rate_GDP}}

Here we see a somewhat clear correlation between growth rate of the GDP growth rate and unemployment rate of the US economy.
As seen, there is an inverse correlation between the two, meaning that when one decreases the other increases and vice versa.
If we look at the effect of the unemployment rate on the GDP growth rate in the UK it should mirror this conclusion. \\

\incfigure{figures/United_Kingdom_GDP_Unemployment}{fig:rate_UK}{Unemployment and Growth rate \cite{UK_Unemployment}\cite{UK_Growth_Rate_GDP}}

As shown, a clear inverse correlation between the growth rate of the GDP and unemployment rate can again be concluded in the UK.
This was expected and supported by the economic theory; Okun´s law.
Okun´s law states that the unemployment rate and GDP of a country have an inverse correlation.
Therefore meaning that the unemployment rate would also have an inverse effect on the growth rate of the GDP. \cite{Economics_Okuns_Law}
This correlation goes both ways, meaning that increasing the GDP growth rate of a country would decrease unemployment and increasing the unemployment rate would decrease the GDP growth rate. \\

Now examining the correlation between unemployment rate and wage inflation, we attempt to see if a correlation exists between the unemployment rate and wage inflation in an economy.
According to keynesian Macro economic theory there is a direct correlation:
The general economic trend is that when unemployment is high the supply of labor exceeds the number of jobs available.
So, when unemployment is high; more workers are available then jobs to fill.
Therefore, employers have little incentive to raise wages to attract workers main stagnant or decrease and wage inflation will not occur.
However, when unemployment is low the demand for labor exceeds the number fo jobs available. 
Therefore, employers now have a strong incentive to pay higher wages to attract workers and wages will increase and wage inflation will occur.\cite{Economics_Unemployment_Inflation}
This is all further supported by the keynesian economic theory, the Philips curve.
The Philips curve supports the inverse correlation between unemployment rate and inflation rate.
It states that "A Philips curve illustrates a tradeoff between the unemployment rate and the inflation rate; if one is higher, the other must be lower." \cite{Economics_Philips_Curve}
Therefore, a conclusion that be made that a inverse correlation exists between unemployment and inflation. \\

Secondly, how does unemployment effects the US and UK unemployed prospects of obtaining a job?
To better discuss the effect of unemployment let us divide unemployment into two time frames, short-term and long-term unemployment.
Short-term unemployment is any unemployment period from 0 to 27 weeks, while long-term unemployment is unemployment 27 weeks or more.
If a correlation can be found between the prospects of obtaining a job and unemployment, a conclusion can then be made on the effect of unemployment on the unemployed prospects of obtaining a job.
Previously, unemployment rate was used to conclude a correlation between unemployment and the economy of the US and UK.
However, it is problematic to make a concrete correlation between the unemployment rate and the prospects of obtaining a job for the unemployed.
This is because, as already discussed, there exists a clear correlation between unemployment rate and the economy of a country.
Therefore, it is difficult to isolate unemployment rate as an independent factor in an experiment, as the unemployment rate is dependent on the economy of a country and vice versa.
A correlation between unemployment rate and the US and UK unemployed prospects of obtaining a job would consequently be meaningless.
As such, any conclusions made on the effect of unemployment would have to come from the status of being unemployed, not the unemployment rate.
So, is there a correlation between being unemployed and the prospects of obtaining a job in the US and UK?
Studies show a clear drop in the prospects of obtaining a job for the unemployed the longer they are unemployed.
This can be seen in the following figure, where the Federal Reserve Bank of New York conducted a study comparing the probability of finding a job and the unemployment duration in months.
The blue line representing when controlling for characteristics such as age, education, ethnicity and gender bias, while the red represents when not taking into account characteristics.
Simply put, the study tries to isolate unemployment as a factor, making the blue line or the full controls group more reliable: \\

\incfigure{figures/Unemployment_Time}{fig:Unemployment_effect_on_unemployed}{Unemployment and Growth rate \cite{Unemployment_effect_unemployed}}

From the figure it can be seen that in the first eight months of being unemployed the job-finding rate falls by roughly 50\% for the controlled group and even lower for the group with no controls.
From this we can conclude that a correlation between being unemployed and the prospects of obtaining a job exists and this correlation is effected by time.
We can similarly conclude that short-term unemployed have a rapidly decreasing probability of obtaining a job, while long-term unemployed have a stable but lowest probability of obtaining a job. 
However, it is productive to ask why this sharp decline in obtaining a job for the unemployed exists.
While many reasons could exist, a major factor is hiring personnel discrimination against unemployed.
A study where researchers sent out the same CV only varying the length of time being unemployed found that "long-term unemployed workers can be up to 45 percent less likely to receive interview invitations than newly unemployed or currently employed people who look just like them".\cite{Unemployment_effect_unemployed}
This indicates that hiring personnel discrimination becomes a major factor the longer the duration of unemployment.
However, this discrimination is not always unwarranted.
A 2018 report by the National Bureau of Economic Research on discrimination of unemployed in relation to time unemployed found that hiring personnel do not discriminate due to the status of being unemployed:\cite{Unemployment_Discrimination}
"We show that such instances are rare when firms discriminate in anticipation of an ultimately unsuccessful application. Discrimination in callbacks is thus largely a response to dynamic selection,".
Dynamic selection referring to a finding of the study that "low ability workers are more likely to be long term unemployed and duration contains information about a worker’s ability".
Therefore, hiring personnel discrimination is based on the likelihood of unemployed being less qualified, meaning that the unemployment status is not what sanctions discrimination, it is a tendency for unemployed to be less qualified.
In conclusion, unemployment has a definitive negative effect on the prospects of the US and UK unemployed prospects of obtaining a job.
These negative effects include that the probability of obtaining a job rapidly decreases for the short-term unemployed and is consistently low for the long-term unemployed.
A negative effect that is hard to deal with is the warranted discrimination of hiring personnel against all unemployed, intensifying with duration of unemployment. \\

In overall conclusion, the original narrowed question posed was:
How does unemployment effect the US and UK economies and there unemployed prospects of obtaining a job?
It is possible to now answer that in terms of unemployment effect on the US and Uk economy, unemployment has an inverse effect on the economies.
This indicates that it is in the best interest of every country to decrease the number of unemployed as this would increase economic productivity and that a.
Thereafter, it can be answered that unemployment decreases the unemployed probability of obtaining a job.
This decrease happens rapidly roughly the first six months and it is in the best interests of any country to urgently find employment for the unemployed. \\

\subsubsection{Obtaining a job and the process of applying for a job}
As discussed, it can become increasingly more difficult for an unemployed to get a job as time goes on.
This overlooks the discussion of the actual process of obtaining a job both for the unemployed and any other individual.
To discuss the process of getting a job, it is productive to first discuss the method by which jobs are obtained.
Thereafter, the process of applying for a job and the time frame of the process of obtaining a job.
So, how does one obtain a job and what does the process of applying for a job look like? \\

Firstly, how do people obtain jobs?
Lou Adler, CEO of "Performance-based hiring learning systems", tried answering just this; he conducted an online survey on LinkedIn and asked how people obtained there current job.
He divided people based on job status before obtaining there current job, dividing people as variations of active or passive job searchers.
Active job searchers are as discussed in background(What is unemployment and how does it materialize) people tht have applied to a job in thr prior four weeks.
He further divided active job searchers into previously unemployed and employed active searchers.
Passive job searchers are therefore anybody who has not applied to a job in the prior four weeks.
He further divided passive job searchers into tiptoers, those who passively search for other jobs without actively searching, and employed who were passively offered a job.
The results of the survey are as followed: \\

The results are outlined in \vref{fig:hiring}.
\incfigure{figures/hiringpeople}{fig:hiring}{How people get jobs 2015 and 2016 \cite{Networking}}
\newpage
Here we see four categories that people could be hired from. \\
The two most relevant categories are; "Apply" which would most likely be an application through CV and "Networking" which would be hiring through acquaintances or media.
From the result we can discern that networking is clearly the most likely factor in obtaining a job, followed by applying in most categories.
However, we see a sizable difference between the effectivity of applying between the active and passive job searchers. 
It is clear that applying has less importance for passive job searchers, while applying has nearly the same effectivity as networking for active job searchers. 
This means that a effective CV process could be as beneficial for active job searchers as good networking.
But what does the process of applying to a job look like for active searchers? \\

The process of applying for a job has five main stages.
These five stages are most relevant to those who apply to a firm through CV, the most likely process of application.
Although every application process is different, most firm/organizations contain these five stages.
The five stages are usually chronological and all have a obstacle that if passed lead to the next stage, ending with a job offer:
\begin{enumerate}
   \item Applying through CV to job position.
   \item ATS scanner scans your CV.
   \item Hiring personnel view you CV.
   \item Job interview with firm/organization.
   \item Job offer.\cite{Process_steps_unemployment}
\end{enumerate}
Each stage merits a discussion of the process it presents..
The first stage "Applying through CV to job position" is when a applicant sends a CV to a firm/organization.
Thereafter the CV is scrutinized by the the second and third stage and is either rejected or accepted to further stages.
The second stage is the ATS scanner and it scans an applicants CV for relevant keywords based on the job position.
The third stage is the hiring personnel who will review the applicants CV and decide which are qualified for a job interview.
The fourth stage is when the applicant is invited to a job interview, possibly more then one interview, the job/organization decides if applicant is compatible.
In the fifth and final stage, if all previous stages are passed, the applicant if given a job offer. \\

But, how long can it take to get a job and how many CV´s does it take?
It is surprisingly difficult to obtain a job though applying.
A survey done by TalentWorks found that on average, when you send out an application, there is an 8.3 percent probability, that you will be invited to a job interview. 
Furthermore, on average, it takes around 10-15 interviews, before one gets a job offer. 
However this greatly varies depending on the person, country and economy.
However, we can conclude that on average it will take us roughly 150 applications before an applicant gets a minimum of one job offer.\cite{HR-sales}
This is a large amount of CV´s and can take a large amount of time to write a CV, apply to jobs and go to job interviews.
Another factor is the time that hiring personnel take before responding to a CV application.
For most jobs it takes around 3 days before the applicant gets a response, however for less demanded job it can take anywhere from 10 to 30 days.\cite{HR-sales}
On average across all jobs, one can expect to hear back from employers withing a week 41 percent of the time and within a couple of weeks 85\% of the time.
This means that the job process can take much longer then anticipated in most cases.
A study conducted by recruitment agency Randstad US surveyed 2000 Americans and found that on average it took five months to obtain a job.\cite{5_month_for_a_job}
Therefore for the average applicant, the process of applying for a job can take a lot of effort and time.
This study was not exclusively conducted on the unemployed, but on a mix of active searchers.
As discussed it can take even longer for the unemployed to obtain a job, especially the long-term unemployed.
Therefore while 5 months is a good lower bounds estimate, it can easily take more then 5 months for the unemployed. \\

\subsubsection{ATS scanner and hiring personnel}
As discussed the process of applying for a job has five stages.
Understanding the five stages in the process of applying to a job is crucial for those applying.

Firstly, the ATS scanner or "Applicant Tracking System", is a keyword scanner that helps hiring personnel screen applicants and narrow the potential candidate pool.
The ATS scanner is widely used and 98 percent of fortune 500 companies use an ATS scanner.
According to Columbia University, 75 percent of applicants are phased out because there CV does not pass the ATS Scanner test.
This is a staggering amount and therefore optimizing ones CV to pass the ATS scanner is vital.
Furthermore, ATS scanners are by no means perfect in there screening of applicants.
A joint Career Arc/Future Workplace Survey found that roughly 62 percent of companies that use an ATS scanner admit that "some qualified candidates are likely being automatically filtered out of the vetting process by mistake".
This can be frustrating for applicants and therefore having a optimized CV for the ATS scanner is the most secure method to passing the ATS scanner.
So, how does the ATS scanner screen an applicants CV and how does an applicant optimize there CV for the ATS scanner?

For an applicant to optimize there CV for the ATS scanner, the way the ATS scans a CV must be discussed.
Many ATS scanners also exist, with the most widely used being the iCIMS, Bullhorn, Taleo, and Greenhouse.
While no two are exactly the same, they all scan for the same underlying elements in an applicants CV:
\begin{enumerate}
   \item Scan education
   \item Scan work experience
   \item Scan relevant keywords throughout CV
\end{enumerate}
However, how does the ATS scanner actually scan for these elements in a CV?
All ATS scanners have slightly different methods, so therefore it is productive to use Taleo, the most widespread ATS scanner as a template example.
Taleos ATS scanner works by having a pre-defined keyword that it searches for, including any relevant information around the keyword.
If a firm/organization is searching for a candidate with a Bachelor in software that has leader characteristics, they will have "Bachelor", "Software" and "leader" as a pre-defined keyword.
Taleo will then search through all applicants CV´s for the keywords "Bachelor", "Software" "leader".
The firm/organization can set the importance of certain pre-defined keywords through Taleo.
Some world maybe be a 
 giving each keyword a rating, most often a system of points.


The ATS scanner will usually first scan for the qualifications of the applicant.
This will always include a scan of education and work experience. 
The ATS scanner will 




James reed book puts much emphasize on the accessibility to content, relevance and wording of a CV. \cite{7_second_test}
The relevance issue could be solved by the same means as the ATS scanner, by implementing keywords that keep the CV specialized for the position.
The accessibility to content could be solved by a structure that invites further attention from the hiring personnel.
The wording of a CV is a difficult task for a program as the situational awareness of context is difficult for a program to solve.
\cite{7_Seconds_to_Get_a_Recruiter_Attention}
\cite{ATS-scanner}
\cite{7_second_test} \\


When applying to a job by means of CV, different job positions advise different required and situational content for the CV.
As outlined in background(Different expectations in a company), over 90 percent of all positions in the digital world will have a screening process.
Most firms will do this through an "Applicant Tracking System" (ATS), which will scan the CV for keywords that match the applicant to the position.
The more keyword matches the CV has, the higher likelihood of the ATS scanner not rejecting the CV.
Furthermore, according to Caitlin Proctor it is stated that "Nearly 75 percent of resumes are rejected because they’re not correctly formatted or keyword optimized."
This means that incorporating keywords relevant to the job and thereafter structuring them correctly in the CV are essential for passing the ATS scanner.
After a CV has passed the ATS scanner it will usually end up in the hands of the respective firms hiring personnel. 
They will be responsible for choosing which CV´s will warrant a job interview. \\

\subsubsection{Optimize ones interview rate}
As discussed, a job interview is most often the final challenge in the application process to obtaining a job. 
An applicants CV also has the highest chance of being rejected before the job interview stage.
Therefore focusing on optimizing a CV to pass the first three stages is extremely useful to securing a job interview and thereafter a job.
So, how does an applicant optimize there chances of being invited to a job interview? \\

Many factors play a role in the chances an applicant has at a job interview.
Including some factors that an applicant has no control over.
To gain a wider understanding all factors will be listed and after categorized as relevant or not. \\

\begin{enumerate}
   \item Monday is the best day to apply for a job, increasing your chances for a job interview by 46 percent more then other days average chances.
   \item If an applicant applies after 10AM the interview chances drop by 5 percent.
   \item Being a woman increases the chances by 48 percent.
   \item Being older, but no older than 35, increases the interview rate by 25 percent.
   \item Having more than one degree, increases ones chances by 22 percent.
   \item Adding industry buzzwords increases your chances by 29 percent.
      e.g. If you are a software developer, then add buzzwords such as machine learning,
      artificial intelligence etc.
   \item Demonstrating earlier job results using numbers increases chances by 40 percent.
      e.g. "Increased profits by 20 percent from Q3 to Q4"
   \item Listing achievements, where you weren't in charge, but only a helping hand
    decreases your chances by 50 percent
      e.g. "Helped management organize financial reports" instead of "Organized financial reports"
   \item Using leadership affiliated buzzwords increases your chances by 51 percent
   \item Not using personal pronouns in the employment section increases your
   chances by 55 percent.
   \item Including a key skills section and buzzwords of the key skills increases your
    chances by 59 percent
   \item Start ones sentences with distinct action verbs, increases ones chances by 140 percent.
      e.g. Do "Developed a mainframe architecture that dramatically increased efficiency"
      instead of "After surveying people, the mainframe architecture that increases efficiency was
      developed by me."\cite{Science_job}
   \end{enumerate}

\subsection{Required and situational content of a CV}
A CV is “a short account of one’s career of qualifications prepared typically by an applicant for a position”.\cite{Difference_between_resume_and_curriculum_Vitae}
When applying for a position within a firm, some aspects of a CV are required. \\
These requirements can come from the position or firm that the CV is intended or from the definition of the CV itself.
Putting aside the firms or positions requirements, all CVs must include the five following requirements to be defined as a CV:
\begin{itemize}
   \item 1. Contact information
   \item 2. CV objective
   \item 3. Relevant skills
   \item 4. Work experience
   \item 5. Education\cite{Write_a_curriculum_Vitae} \\
\end{itemize}
The substance of each requirement varies from applicant to applicant. However, every CV must include these five requirements to be effective.
A further explanation of each requirement is due:
Contact information is required as the firm at the bare minimum must have some way to contact the person so he or she can be accepted for the position.
CV objective is required as it specifies what and who the CV is intended and without it the CV can fill no purpose.
Relevant skills are required as without it you have no relation to the CV objective.
Work experience and education is required as without it one has no qualifications. \\

As a CV is an account of one’s qualifications it is also possible to leave education and work experience empty if one has none.
However it is then hardly an effective CV.\cite{Difference_between_resume_and_curriculum_Vitae}
Along with the requirements of CV there are also aspects which are situational.
These vary and most likely the firm or position in which one is applying to will lay out these aspects, or it is apparent from the position itself.
If not specified it is difficult to know what to include. To little and one will under-qualified, to much and the CV will be to long.\cite{Job_Application_for_science}, excesses and unorganized.
Therefore, it is essential to distinguish one’s CV with the right quantity and quality of situational content.
Lets examine some of the more typical situational content that could be included in a CV: \\
\begin{itemize}
   \item 1. Professional association
   \item 2. Volunteer experience
   \item 3. Languages
   \item 4. Additional training courses
   \item 5. Publication
   \item 6. Awards/Honors
   \item 7. References\cite{6_sections} \\
\end{itemize}
The effectiveness of a CV can drastically change due to use of situational content.
Therefore it is necessary to further explain each situational content: \\
Professional association is any trade unions, learned societies, regulatory universities and other inter-professional societies.
Many associations have certain prestiges and hold there member to a certain standard of quality.
Therefore a professional association can improve a CV if relevant.\cite{Professional_associations_and_organizations}\cite{Perks_of_professional_organizations}
Volunteer experience is any volunteer work relevant to the position.
Languages is any spoken or written language relevant to the position.
As most firms in our interconnected market interact with some multilingualism, languages can easily increase the quality of a CV.
Additional training courses are any extra courses relevant to the position. 
Publication are any reports, books or other published materials that could show qualifications for the given position.
Awards and honors are any university or professional awards or honors given that show qualification for thr given position.
References are very situational, as putting references in a CV may make you seem unsure of yourself and in need of validation from others to show qualifications.
However, if a CV has the right references it can assure employers of your qualifications and past experience.\\

\subsection{Structures of the CV}
Writing an effective isn´s just about including the right content. 
Its also about how you present that information.
Every applicant must have a structure that best presents there CV.
It is therefore relevant to discuss the different structures available to the applicant. 
In theory, an applicant could structure there content a million different ways.
To narrow it down we will not discuss the three most common structures and the pros and cons of each of them.

The first structure is the Chronological CV structure: \\
\begin{itemize}
   \item  Contact information
   \item  CV objective
   \item  Work experience
   \item  Education
   \item  Relevant skills
   \end{itemize}
The Chronological CV structure is the most common structure and gives an easy overview over the applicants content.
It is adaptable and features all required content mentioned in (Required and situational content of a CV).
With the added section "relevant skills" where the applicant can include there situational content to standout.
Its traditional structure is simple for both hiring personnel and the ATS Scanner to view.
The structure is best suited for those who are confident in there CV content, especially there work experience and education.
The structure can have a difficult standing out as it is the most common and should not be used by those who have big gaps in employment.\\

The second structure is the Functional CV structure:
\begin{itemize}
   \item  Contact information
   \item  CV objective
   \item  Relevant skills
   \item  Situational content
   \item  Work experience
   \item  Education\\ 
   \end{itemize}
The Functional CV structure focuses on relevant skills and situational content over work experience.
This structure is favoured by those applicants with large gaps in there unemployment history, or in the middle of a career change.
As the structure emphasizes the applicants relevant skills and situational content it is flexible as the applicant has a chance to present himself on his own terms.
This structure is highly dependant on the applicants relevant skills and how they present there significance to the job they are applying.
The structure is highly flexible and can either be a nightmare for both the hiring personnel and ATS scanner, or if structured correctly with correct usage of keywords be highly effective.\\
 
The third structure is the Combination CV structure:
\begin{itemize}
   \item  Contact information
   \item  CV objective
   \item  Relevant skills
   \item  Work experience
   \item  Situational content
   \item  Education \\
   \end{itemize}
The Combination CV structure is as stated, a combination of the Chronological and Functional CV structures. 
It enables the applicant with an otherwise mediocre content to combine the two and compensate a weakness with another CV structure.
This is a highly flexible structure and lets an applicant who knows how to present oneself to standout.
It is not a structure that an applicant with either overwhelming relevant content or no relevant content to use.
As the structure does not allow for a high emphasize on one section without an obvious lack in the others.
   \cite{Resume_structure}
   \cite{Tips_for_best_format}
   \cite{8_Best_cv_format}
\clearpage
