\section{Background}\label{sec:background}



\subsection{Background Data}

\subsubsection{arbejdsløshed -- Kristoffer}


\subsubsection{The numbers behind the process of getting a job}
How did people get their current job?
Lou Adler tried answering just this: He conducted an online survey 
on linkedin based on 3000 answers, where in most of these answers 
came from those actually hiring. 

The results are outlined in the illustration underneath:
\includegraphics{figures/hiringpeople.bmp}
https://www.linkedin.com/pulse/new-survey-reveals-85-all-jobs-filled-via-networking-lou-adler/

Here we can see, that active candidates only represent 15-20 percent of the
entire jobmarket. Around 15-20 percent are only tiptoeing around the idea 
of getting a new job, while the rest are passive candidates (meaning people who are 
satisfied in their current position.).

Based on these graphs, we can see that at the very minimum, 42 percent of people are 
hired based on networking, and that is only if you already have a job. 
Whilst this percentage only goes up, the oppertunity for a job seeking person
to get hired based on an application only goes down. This reflects the 
most effective way to get hired is through internal applications or networking.  
On the flip side, it also nicely illustrates just how stiff competition there actually is,
if your only way of getting hired is through sending applications.

When you send out an application, there is an 8.3 percent probability, that
they will actually invite you to a job interview. Furthermore it takes around 
10-15 interviews, before one gets a job offer. Obviously it varies depending
on educational background, job type and many other factors, but this is the average.
Some quick math ( (((100/8,3)*10)+((100/8,3)*15))/2 = 150..) tells us, that it will 
take an average of 150 ish applications before one gets a job offer.
https://talent.works/2017/09/22/how-long-does-it-take-to-get-a-job-60-days-if-youre-in-hr-or-sales/

All these applications add up, before one can reap the reward. According to a 
study conducted surveying 2000 Americans, by recruitment agency Randstad US 
discovering the “art of the job hunt”, it takes an average of five
months from when the job search begins, until one actually lands
the job. 
https://www.swnsdigital.com/2018/10/it-takes-5-months-of-searching-to-land-a-job-study-finds/ 

To add insult to injurry, after all the hard work of creating an application, it can 
take quite some time before the hiring managers actually respond, that is if they ever 
bother answering your application to begin with.
It takes around 3 days between they receive the application, before they answer.
This is the case for the most in demand roles in society, for the less in demand roles 
such as writers, nurses and unskilled labour, it can be from 10 to over 30 days.
https://talent.works/2017/09/22/how-long-does-it-take-to-get-a-job-60-days-if-youre-in-hr-or-sales/
On average one can expect to hear back from employers within a week 41 percent
of the time. Within a couple of weeks 85 percent of the time. 
https://www.indeed.com/career-advice/finding-a-job/how-long-should-you-wait-to-hear-back-about-a-job

Below some of the more popular jobs are illustrated as a function of interview 
rate on the left and response delay on the left:  
\includegraphics{figures/interviewratexresponse delay.bmp}
https://talent.works/2017/09/22/how-long-does-it-take-to-get-a-job-60-days-if-youre-in-hr-or-sales/
The interview rate is further supported from a danish online survey, that concluded
that 65 percent of people get an interview within the first 15 applications and
 82.5 percent of people get an interview within the first 30 applications.
 https://get2business.wordpress.com/2009/10/27/hvor-mange-ans%C3%B8gninger-skal-der-til-for-at-fa-et-job/

\subsubsection{Optimize ones interview rate}
There are many factors to consider, if one wishes to optimize ones 
chances of getting an interview, one of the more empirical proven ones
is what time and day one sends their application. 
To get the highest chances, you have to apply between early Tuesday morning
and Thursday before noon using the employers local time. Monday is even better, 
increasing your chances by 46 percent in regard to the average. 
If one should apply on another day, the most important factor is that
it's done before 10AM, since the interview chances drops below 5 percent for 
the majority of late evening pplications.
https://insights.dice.com/2019/10/30/best-times-days-submit-resume/

Perhaps one of the most influential condition of weither you get an interview, 
is how fast you are at applying:
Based on 30000 datapoints from the company Speedrecruiters, you need
to apply within the first 14 days to have a practical chance of getting an
interview. it is such that 50 percent of the people who got an interview
for the job applied within the first week, and 75 percent of those who
got an interview applied within the first 14 days, whilst the chancess of 
getting an interview thereafter dwindles exponentially.
https://www.jobfinder.dk/artikel/her-bedste-tidspunkt-at-sende-din-ansoegning/220987

Other factors that influenze the interview rate are as following:
1. Being a woman increases the chances by 48 percent.
2. Being older, but no older than 35, increases the interview rate by 25 percent.
3. Having more than one degree, increases ones chances by 22 percent.
4. Adding industry buzzwords increases your chances by 29 percent.
5. Demonstrate earlier job results using numbers increases chances by 40 percent.
6. Listing achievements, where you weren't in charge, but only a helping hand
 decreases your chances by 50 percent
7. Using leadership affiliated buzzwords increases your chances by 51 percent
8. Not using personal pronouns in the employment section increases your
chances by 55 percent.
9. Including a key skills section and buzzwords of the key skills increases your
 chances by 59 percent
10. Start ones sentences with distinct action verbs, increases ones chances by 140 percent.
Ptalent.works/2018/01/08/the-science-of-the-job-search-part-i-13-data-backed-ways-to-win/


%how long does it take to actually get a job

%how do these people then end up getting the job

%for the people sending applications, how many did they sen? ++ details

%best time to send applications, and why it can benefit people who
%dont have time there ++ howl ong should you wait with a followup
